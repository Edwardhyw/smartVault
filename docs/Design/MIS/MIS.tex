\documentclass[12pt, titlepage]{article}

\usepackage{amsmath, mathtools}

\usepackage[round]{natbib}
\usepackage{amsfonts}
\usepackage{amssymb}
\usepackage{graphicx}
\usepackage{colortbl}
\usepackage{xr}
\usepackage{hyperref}
\usepackage{longtable}
\usepackage{xfrac}
\usepackage{tabularx}
\usepackage{float}
\usepackage{siunitx}
\usepackage{booktabs}
\usepackage{multirow}
\usepackage[section]{placeins}
\usepackage{caption}
\usepackage{fullpage}

\hypersetup{
bookmarks=true,     % show bookmarks bar?
colorlinks=true,       % false: boxed links; true: colored links
linkcolor=red,          % color of internal links (change box color with linkbordercolor)
citecolor=blue,      % color of links to bibliography
filecolor=magenta,  % color of file links
urlcolor=cyan          % color of external links
}

\usepackage{array}

%\externaldocument{../../SRS/SRS}

\input{../../Comments}
%% Common Parts

\newcommand{\progname}{Mechtronics Enigeering} % PUT YOUR PROGRAM NAME HERE
\newcommand{\authname}{Team 32, Wingman
\\ Edward He
\\ Erping Zhang
\\ Guangwei Tang
\\ Peng Cui
\\ Peihua Jin } % AUTHOR NAMES                  

\usepackage{hyperref}
    \hypersetup{colorlinks=true, linkcolor=blue, citecolor=blue, filecolor=blue,
                urlcolor=blue, unicode=false}
    \urlstyle{same}
                                


\begin{document}

\title{Module Interface Specification for \progname{}}

\author{\authname}

\date{\today}

\maketitle

\pagenumbering{roman}

\section{Revision History}

\begin{tabularx}{\textwidth}{p{2cm}p{6cm}X}
\toprule {\bf Date} & {\bf Version} & {\bf Authors}\\
\midrule
2023-01-18 & 1.0 & Edward He, Erping Zhang, Guangwei Tang, Peng Cui, Peihua Jin\\\\
2023-01-30 & 2.0 Update MIS of Login Module and Information Exxtraction Module & Peng Cui\\\\

2023-01-30 & 2.1 Update MIS of Communication Port 1 Module and Communication Port 2 Module & Erping Zhang, Guangwei Tang\\\\

2023-02-01 & 2.2 Update File Storage Module and Image Processing Module & Edward He, Peihua Jin\\\\

2023-03-10 & 2.3 Update Motor Control Module & Erping Zhang\\\\

2023-04-04 & 3.0 Check for grammar and structure & Peng Cui\\
\bottomrule
\end{tabularx}

~\newpage

\section{Symbols, Abbreviations and Acronyms}

See SRS Documentation at \url{https://github.com/Edwardhyw/smartVault/tree/main/docs/SRS}


\newpage

\tableofcontents

\newpage

\pagenumbering{arabic}

\section{Introduction}

The following document details the Module Interface Specifications for SmartVault, a Mechatronics system that aims to assist users in finding their belongings. The functions and variables used in this project will be specified. The complementary documents include the System Requirement Specifications and Module Guide. The full documentation and implementation can be
found at \url{https://github.com/Edwardhyw/smartVault}.  

\section{Notation}
 The following table includes the notation we will need in the motor control module.

\begin{center}
\renewcommand{\arraystretch}{1.2}
\noindent 
\begin{tabular}{l l p{7.5cm}} 
\toprule 
\textbf{Data Type} & \textbf{Notation} & \textbf{Description}\\ 
\midrule
float & \tau & The time gap between each motor step motion\\
float & \omega & The angular velocity of the motor movement\\
float & \theta & The angule degree of the motor movement\\

\bottomrule
\end{tabular} 
\end{center}

\noindent


\section{Module Decomposition}

The Whole project is decomposed into two main parts: Hardware Module and Software Module. The Hardware Module is mainly designed for motor control and Hardware-Software Communication. The Software Module is designed for main human interface. It is the main part of the design of the project. These two modules are further divided into seven parts. The first part of the decomposition is Communication Port 2 which is used for sending signals to the software part of the project. The second decomposition of the Hardware Module is Motor Control, it is used to control the real-time position of the camera so that the human body is within the centre of the screen taken by the camera.\\ 
When it comes to the Software Module, The first part is Login Module, it is designed ask the user log into the running program. It can also be treated as the first barrier of protection of the user's privacy. If the person does not enter the correct username and password, he or she cannot enter the search window, which means the private information is blocked from the user. The second part is the Information Storage Module. It is used to store the screenshot of the images taken by the camera, which shows the position information of the object. The third part is Image Processing Module. It helps analyze the image to get the key information that is useful to the user. Currently this module contains the human body detection method and object movement method. The forth part is  Information Extraction Module. It is related to the Information Storage Module and is used to choose pictures from the database which meets the information required by the user. The fifth part decomposed from the Software Module is Communication Port 1. It can be treated as the connection bridge between the hardware and software. 

\begin{table}[H]
\centering
\begin{tabular}{p{0.3\textwidth} p{0.6\textwidth}}
\toprule
\textbf{Level 1} & \textbf{Level 2}\\
\midrule

\multirow{5}{0.3\textwidth}{Software-Module} & 
Login  \\
& Information Storage \\
& Image Processing \\
& Information Extraction \\
& Communication Port 1 \\
\midrule

\multirow{2}{0.3\textwidth}{Hardware-module} & Communication Port 2\\
& Motor Control\\


\bottomrule

\end{tabular}
\caption{Module Hierarchy}
\label{TblMH}
\end{table}

\newpage

\section{MIS of Login Module} 

Responsible for interacting with the user. The user can only start the program successfully with correct username and password. It can also provide technical support information to the user.

\subsection{Uses}
The Login Module uses python tkinter library to creates different windows, input boxes, buttons, and so on. It also interacts with the Information Extraction Module for the user to find their desired object location. 

\subsection{Syntax}

\subsubsection{Constants}


\begin{table}[H]
\caption{Constants used in Login Module} 
\begin{tabularx}{\textwidth}{XX}
\toprule
\textbf{Name of Constants} & \textbf{Value}\\
\midrule
Username & Initially determined by the user\\\\
Password & Initially determined by the user\\\\
Technical Support Window Geometry & Initially 450*300\\\\
Login Window Geometry & Initially 450*300\\\\
Location & Initial file storage location determined by the user\\\\
Software Developer Contact Information & Email of members of this project\\

\bottomrule
\end{tabularx}
\end{table}

\subsubsection{Access Programs}
\begin{center}
\begin{tabular}{|p{2cm}|p{2cm}|l|p{6cm}|}
\hline
\textbf{Name} & \textbf{Input} & \textbf{Output} & \textbf{Description}\\
\hline
Tech & N/A & Technical Support Window & To present the technical support window to the user.\\
\hline
Input & N/A & N/A & To check the correctness of the input username and password. If both are correct, then the Information Extraction will be called.\\
\hline
content & N/A & N/A & To fill the Login Window with different buttons, input boxes, and texts. \\
\hline

\end{tabular}
\end{center}

\subsection{Semantics}

\subsubsection{State Variables}

\begin{table}[H]
\caption{State Variables used in Login Module} 
\begin{tabularx}{\textwidth}{XX}
\toprule
\textbf{Name of State Variables} & \textbf{Description}\\
\midrule
self.win & Welcome window itself\\
self.title & The title of the window\\
self.size & The size of this window\\
self.user\_input & Username input box\\
self.pass\_input & Password input box\\
self.location & File storage location\\

\bottomrule
\end{tabularx}
\end{table}

\subsubsection{Environment Variables}

\begin{center}
\begin{tabular}{|p{8cm}|p{6cm}|}
\hline
\textbf{Environment Variables} & \textbf{Description}\\
\hline
Input Username & The input value in Username Box\\
\hline
Input Password & The input value in Password Box\\
\hline

\end{tabular}
\end{center}

\subsubsection{Assumptions}

N/A

\subsubsection{Access Routine Semantics}

\noindent Tech():
\begin{itemize} 
\item transition: Create a new window and sees self.win as the toplevel window.
\item output: N/A
\item exception: N/A 
\end{itemize}
\noindent Input():
\begin{itemize} 
\item transition: Updating and then checking self.user\_input and self.pass\_input.
\item output: If the accessing to information Extraction Module is successful.
\item exception: Warning text will be shown if the username or password is wrong.
\end{itemize}
\noindent content():
\begin{itemize} 
\item transition: Use self.title and self.size inside the self.win to create the welcome window.
\item output: The output welcome window.
\item exception: N/A 
\end{itemize}

\subsubsection{Local Functions}

N/A

\newpage

\section{MIS of File Storage Module} 
Responsible for storing the desired frames or photos into certain directory. The module mainly consists of two functions, createFolder() and frameStorage(), which are used to initialize the basic directory with folders and save the frames respectively.
\subsection{Uses}
The File Storage Module imports OpenCV, OS, datetime, numpy libraries. It also communicates with the Image Processing Module in order to transmit and save the desired data. Additionally, it interacts with Information Extraction Module to get access into User Interface and collect information from the main window.\\
\begin{itemize}
\item OpenCV: Writing in function\\
\item OS: Mainly used to process directory and path\\
\item datatime: Serve for fetching date and time\\
\item numpy: Used to create empty array\\
\end{itemize}
\subsection{Syntax}

\subsubsection{Constants}
\begin{table}[H]
\caption{Constants used in File Storage Module} 
\begin{tabularx}{\textwidth}{XX}
\toprule
\textbf{Name of Constants} & \textbf{Value}\\
\midrule
img\_item & 100\\\\
img\_loca & 100\\
\bottomrule
\end{tabularx}
\end{table}
\subsubsection{Access Programs}
\begin{center}
\begin{tabular}{|p{3cm}|p{2cm}|l|p{6cm}|}
\hline
\textbf{Name} & \textbf{Input} & \textbf{Output} & \textbf{Description}\\
\hline
createFolder & directory & N/A & To create 3 basic folders under the desired directory\\
\hline
frameStorage & directory, itemNum, item, location, key & N/A & To save the frames\\
\hline

\end{tabular}
\end{center}
\subsection{Semantics}

\subsubsection{State Variables}
\begin{table}[H]
\caption{State Variables used in File Storage Module} 
\begin{tabularx}{\textwidth}{XX}
\toprule
\textbf{Name of State Variables} & \textbf{Description}\\
\midrule
directory & paths for creating folders\\
fileName & The name of saved frames\\
current_time & To get the actual time\\
path1 & path for saving frames into 'item' folder\\
path2 & path for saving frames into 'location' folder\\
img\_item & image list for 'item' folder\\
img\_loca & image list for 'location' folder\\
itemNum & image inside 'location' folder\\
item & image inside 'item' folder\\
location & image inside 'location' folder\\
key & choose whether to insert new item or update recorded items\\
\bottomrule
\end{tabularx}
\end{table}
\subsubsection{Environment Variables}
N/A
\subsubsection{Assumptions}
The frames sent from Image Processing Module are assumed to be correct all the time. The File Storage Module itself has no ability to distinguish whether the data are expected.

\subsubsection{Access Routine Semantics}

\noindent createFolder():
\begin{itemize} 
\item transition: Try to create a main folder with 'item' and 'location'. If these three folders have already built, then do nothing.
\item output: N/A
\item exception: If the directory is invalid, do nothing.
\end{itemize}
\noindent frameStorage():
\begin{itemize} 
\item transition: To process the transmitted data or frames through the input parameters of keys. For the insertion of new item, write into both 'item' folder and 'location' folder. For the updating of recorded items, write into 'location' folder only.
\item output: Frames will be sent to correct folders
\item exception: N/A
\end{itemize}

\subsubsection{Local Functions}

N/A

\newpage

\section{MIS of Information Extraction Module} 

Responsible for extracting information from the File Storage Module based on the inputs given by the user.

\subsection{Uses}

The Information Extraction Module python tkinter library to access some window manipulations. It also uses python datetime and PIL libraries to extract images based in the timing input given by the user. It will only extract pictures from the File Storage Module nased on the file path initialized at the start of the program

\subsection{Syntax}

\subsubsection{Constants}


\begin{table}[H]
\caption{Constants used in Information Extraction Module} 
\begin{tabularx}{\textwidth}{XX}
\toprule
\textbf{Name of Constants} & \textbf{Value}\\
\midrule
Location & Initial file storage location determined by the user\\\\
Window title and sizes & determined by the software developer\\\\
Time slot choosing slot & 1 day, 7 days, 30 days, 100 days and 365 days\\

\bottomrule
\end{tabularx}
\end{table}

\subsubsection{Access Programs}


\begin{center}
\begin{tabular}{|p{2cm}|p{2.5cm}|l|p{7.5cm}|}
\hline
\textbf{Name} & \textbf{Input} & \textbf{Output} & \textbf{Description}\\
\hline
content & N/A & N/A & Filling the window with different buttons, input boxes or other features.\\
\hline
search & button pressed in the window & search window & Based on the time values selected by the user, it calls the search window.\\
\hline
sea & N/A & N/A & To fill the search Window with different buttons, input boxes, and texts. \\
\hline
nex\_img & N/A & N/A & The window shows the next image suitable for the time value. If it is at the end of the image list, the first image will be presented.\\
\hline
pre\_img & N/A & N/A & The window shows the previous image suitable for the time value. If it is in the first image, the last image will be shown. \\
\hline
find\_img & N/A & N/A & After the user press the Find button, the result find window will be shown. \\
\hline
fin & N/A & N/A & Creating the result window with the location image shown on that window.\\
\hline

\end{tabular}
\end{center}

\subsection{Semantics}

\subsubsection{State Variables}

\begin{table}[H]
\caption{State Variables used in Information Extraction Module} 
\begin{tabularx}{\textwidth}{XX}
\toprule
\textbf{Name of State Variables} & \textbf{Description}\\
\midrule
self.window & The choose window itself\\
self.variable & Selecting box presents in the window\\
self.choose\_list & A list of time variables the user can choose\\
self.loca & File storage location\\
self.sear & The search window itself\\
self.Item\_dir & The directory of the item file\\
self.label & Label used in the window\\
self.image\_sort & Sorting number of the image\\
self.imge\_list & List of images chosen from the item file\\
self.image\_name & The name of the item image\\
self.today & The date of today\\
self.date\_sub & time variable chosen by the user\\
self.name & The name of the image chosen by the user\\
self.find & The result window itself\\
self.Location\_dir & The path of the location file\\
self.image\_sort(find) & The sorting number of location image\\
self.image\_flag & flag for the image\\
self.image\_name(find) & The name of the location image\\

\bottomrule
\end{tabularx}
\end{table}

\subsubsection{Environment Variables}

N/A

\subsubsection{Assumptions}

N/A

\subsubsection{Access Routine Semantics}

\noindent content():
\begin{itemize} 
\item transition: Create a new window based on window state variable.
\item output: A window presented to the user.
\item exception: N/A 
\end{itemize}
\noindent search():
\begin{itemize} 
\item transition: Updating self.variable and pass this variable to self.sear to create the search window.
\item output: The search window
\item exception: N/A
\end{itemize}
\noindent sea():
\begin{itemize} 
\item transition: Create the search window based on title and geometry. It will also select a list of images from the File Storage Module based on the time parameters passed to it and shows the first image.
\item output: The output search window with image shown.
\item exception: Nothing will be shown if the time variable is not applicable. 
\end{itemize}
\noindent next\_img():
\begin{itemize} 
\item transition: Update self.image\_sort by adding 1 and choose the next image from self.image\_list. If the sort number is equal to the length of the list -1, set self.image\_sort to zero.
\item output: The change of image shown in the window.
\item exception: N/A
\end{itemize}
\noindent pre\_img():
\begin{itemize} 
\item transition: Update self.image\_sort by subtracting from 1 and choose the previous image from self.image\_list. If the sort number is equal to zero, set self.image\_sort to length of the list -1.
\item output: The change of image shown in the window.
\item exception: N/A
\end{itemize}
\noindent find\_img():
\begin{itemize} 
\item transition: Destroy the current search window and create the result window and pass the selected image name to that window.
\item output: The change of window shown in the screen.
\item exception: N/A
\end{itemize}
\noindent fin():
\begin{itemize} 
\item transition: Compare self.image\_name(find) with the parameters passed to this window. If the item numbers are the same, present this location image.
\item output: The image shown in the window.
\item exception: N/A
\end{itemize}
\subsubsection{Local Functions}

N/A

\newpage

\section{MIS of Image Processing Module} 

 The Image process module is the main function of the project. It will take two pictures, using the image subtracting method to find the object that is moved over the time interval and pass the result to File Storage Module. 

\subsection{Uses}

\begin{itemize}
\item Opencv library for image processing
\item scikit image library for calculating difference between images
\item File Storage Module
\end{itemize}
\subsection{Syntax}

\subsubsection{Constants}

\begin{table}[H]
\caption{Constants used in Image Processing Module} 
\begin{tabularx}{\textwidth}{XX}
\toprule
\textbf{Name of Constants} & \textbf{Value}\\
\midrule
kernel & 9x9 matrix of value 1\\\\
item\_area & 300\\\\
folder & target folder for storing images\\

\bottomrule
\end{tabularx}
\end{table}



\subsubsection{Access Programs}

\begin{center}
\begin{tabular}{|l|l|l| p{7cm} |}
\hline
\textbf{Name} & \textbf{Input} & \textbf{Output} & \textbf{Description}  \\
\hline
load\_images\_from\_folder & folder & images & load all images from a folder\\
\hline
item\_match & item,img & Boolean True/False & compare the item with image and see if there is an identical item in the image\\
\hline
item\_comp & itemList1, itemList2,img\_list,img & itemList1,loc\_list & sort images from the two item list and put them into itemList and loc\_list accordingly.\\
\hline
extract\_item & path, orimg1,orimg2 & N/A & extracting moving items by getting the difference between the two frames. Applying image processing and central function for the image processing module \\
\hline
item\_store & itemList, positionList, directory & N/A & identify whether item is to be stored or updating its recorded position. Also item storing is happening in this function \\
\hline
begin\_tracking & path & N/A & begin human tracking  \\
\hline
\end{tabular}
\end{center}

\subsection{Semantics}

\subsubsection{State Variables}

\begin{center}
\begin{tabular}{|p{4cm}|p{4cm}|l|}
\hline
\textbf{Name} & \textbf{value} & \textbf{Description}\\
\hline
self.flag & N/A & flag for human detection\\
\hline

\end{tabular}
\end{center}


\subsubsection{Environment Variables}

N/A


\subsubsection{Assumptions}

N/A

\subsubsection{Access Routine Semantics}

\noindent begin\_tracking():
\begin{itemize}
\item transition: If self.flag $>$ 0 
\item output: call extract\_item(), pass two frames for item extraction 
\item exception: N/A
\end{itemize}
\noindent extract\_item():
\begin{itemize}
\item transition: after all items were found in the frame, call item\_store(), call item\_comp
\item output: N/A
\item exception: N/A
\end{itemize}
\noindent item\_store():
\begin{itemize}
\item transition: call load\_images\_from\_folder, check if item\_match is true, call frameStorage(), if flag is 0, also call frameStorage() 
\item output: N/A.
\item exception: N/A
\end{itemize}
\noindent item\_match():
\begin{itemize}
\item transition: N/A
\item output: Boolean expression. If items matched, return true, if not, return false
\item exception: N/A
\end{itemize}
\noindent item\_comp():
\begin{itemize}
\item transition: call item\_match
\item output: list of item and list of location
\item exception: N/A
\end{itemize}
\noindent load\_images\_from\_folder():
\begin{itemize}
\item transition: N/A
\item output: list of images from the storage folder.
\item exception: N/A
\end{itemize}



\subsubsection{Local Functions}

N/A

\newpage

\section{MIS of Communication Port 1 Module} 



\subsection{Uses}
Communication port 1 is a function that runs on laptop, used to send the motor control signal to communication port 2, which is handled by the Arduino board.


\subsubsection{Constants}
\subsection{Syntax}
\begin{table}[H]
\caption{Constants used in Communication Port 1 Module} 
\begin{tabularx}{\textwidth}{XX}
\toprule
\textbf{Name of Constants} & \textbf{Value}\\
\midrule
Time gap & The time delay between each control signal\\\\
Baud rate & The communication baud rate between communication port 1 and port 2\\\\

\bottomrule
\end{tabularx}
\end{table}

\subsubsection{Access Programs}

\begin{center}
\begin{tabular}{|p{2cm}|p{2.5cm}|l|p{7.5cm}|}
\hline
\textbf{Name} & \textbf{Input} & \textbf{Output} & \textbf{Description}\\
\hline
begin\_ tracking & N/A & N/A & Use the result from the human tracking algorithm and send the control signal\\
\hline
Back\_to \_origin & N/A & N/A & Send a special control signal to the Arduino, to make the camera move to the original angle\\
\hline



\end{tabular}
\end{center}

\subsection{Semantics}

\subsubsection{State Variables}

\begin{table}[H]
\caption{State Variables used in Communication Port 1 Module} 
\begin{tabularx}{\textwidth}{XX}
\toprule
\textbf{Name of State Variables} & \textbf{Description}\\
\midrule
self.port & The com port is used for the serial communication\\\\
self.baudrate & The baud rate of the serial communication\\\\
self.servo & The indicator of servo connection\\


\bottomrule
\end{tabularx}
\end{table}

\subsubsection{Environment Variables}
N/A


\subsubsection{Assumptions}

The communication between port 1 and port 2 doesn't have any delay. Once the camera captures the human motion, it will send the signal with no time delay.

\subsubsection{Access Routine Semantics}

\noindent back\_to\_origin:
\begin{itemize}
\item transition: if True  
\item output: Send a special reset signal to Communication Port 2 Module 
\item exception: N/A 
\end{itemize}

\noindent begin\_tracking:
\begin{itemize}
\item transition: N/A
\item output: send signal and data to Communication Port 2 Module 
\item exception: N/A 
\end{itemize}


\subsubsection{Local Functions}

N/A

\newpage


\section{MIS of Communication Port 2 Module} 



\subsection{Uses}
Receive control signal from communication port 1. The program is located in the Arduino control board.

\subsection{Syntax}
\begin{table}[H]
\caption{Constants used in Communication Port 2 Module} 
\begin{tabularx}{\textwidth}{XX}
\toprule
\textbf{Name of Constants} & \textbf{Value}\\
\midrule
Time gap & The time delay between each control signal\\\\
Baud rate & The communication baud rate between communication port 1 and port 2\\\\

\bottomrule
\end{tabularx}
\end{table}


\subsubsection{Constants}
N/A

\subsubsection{Access Programs}

\begin{center}
\begin{tabular}{|p{2cm}|p{2.5cm}|l|p{7.5cm}|}
\hline
\textbf{Name} & \textbf{Input} & \textbf{Output} & \textbf{Description}\\
\hline
serial.print & N/A & N/A & Send a character using serial communication\\
\hline
serial.begin& N/A & N/A & Initialize the serial communication\\
\hline
serial.read& N/A & N/A & Read a character from serial communication \\
\hline
serial.end& N/A & N/A & Turn off the serial communication and clean the buffer\\
\hline



\end{tabular}
\end{center}

\subsection{Semantics}

\subsubsection{State Variables}

N/A

\subsubsection{Environment Variables}

N/A

\subsubsection{Assumptions}

The serial communication can be initialized and end very fast, the time to initialize and end the serial port can be ignored

\subsubsection{Access Routine Semantics}


\noindent serial.print:
\begin{itemize}
\item transition: if True  
\item output: send a character to Communication Port 1 Module 
\item exception: N/A 
\end{itemize}

\noindent serial.begin:
\begin{itemize}
\item transition: if True  
\item output: Initialize the seral communication
\item exception: N/A 
\end{itemize}

\noindent serial.read:
\begin{itemize}
\item transition: if True  
\item output: Return a character from the serial buffer
\item exception: N/A 
\end{itemize}

\noindent serial.end:
\begin{itemize}
\item transition: N/A
\item output: Turn off the serial communication  
\item exception: N/A 
\end{itemize}


\subsubsection{Local Functions}

N/A


\newpage

\section{MIS of Motor Control Module} 



\subsection{Uses}
Get the control signal from communication port 2 and control the servo motor through a PWM pin on the Arduino.

\subsection{Syntax}

\subsubsection{Constants}
\begin{tabular}{|p{0.2\textwidth}|p{0.15\textwidth}|p{0.1\textwidth}|p{0.2\textwidth}|p{0.3\textwidth}|}

\hline \multicolumn{5}{|c|}{Table 3: Constants Variables}\\

\hline Constant Name&Constant Type&Value&Units &Comment\\

\hline Angle per step&float&2&Degree/step&This is the angle movement stepper motor will move after 1 signal \\

\hline Height of the Camera&float&10&mm&This is the distance between the lens of camera and the bottom of the mount\\

\hline Resolution&Integer&1920x1080&Pixel&This is the resolution of the camera\\
\hline Arduino input voltage&float&9.0&V&This is the input voltage of the Arduino board\\


\hline

\end{tabular}


\subsubsection{Access Programs}


\begin{center}
\begin{tabular}{p{4cm} p{8cm} }
\hline
\textbf{Name} & \textbf{Description}  \\
\hline
back\_to\_origin & rotate the camera to the original angle  \\\\

xincrease & rotate the camera in the clockwise horizontal direction for 1 step \\\\

xdecrease & rotate the camera in the horizontal counter-clockwise direction for 1 step \\\\


yincrease & rotate the camera in the vertical clockwise direction for 1 step  \\\\


ydecrease & rotate the camera in the vertical counter-clockwise direction for 1 step  \\




\end{tabular}
\end{center}

\subsection{Semantics}

\subsubsection{State Variables}

\begin{table}[H]
\caption{State Variables used in Motor Control Module} 
\begin{tabularx}{\textwidth}{XX}
\toprule
\textbf{Name of State Variables} & \textbf{Description}\\
\midrule
waittime & The time delay between each step motion, this is for protecting the motor from overload burning\\\\
ang_per_step & The angle for each movement of the servo motor\\\\
origin_ang & The original angle when the system starts \\\\
upper & The upper limit of the angle that the motor can rotate\\\\
lower & The lower limit of the angle that the motor can rotate\\\\


\bottomrule
\end{tabularx}
\end{table}


\subsubsection{Environment Variables}

N/A

\subsubsection{Assumptions}

The servo motor move in a absolute angle, which means each step will move the angle very accurately.

\subsubsection{Access Routine Semantics}

\noindent xincrease():
\begin{itemize}
\item transition: if True 
\item output: Rotate the camera in the clockwise horizontal direction for 1 step 
\item exception: N/A 
\end{itemize}


\noindent xincrease():
\begin{itemize}
\item transition: if True 
\item output: Rotate the camera in the clockwise horizontal direction for 1 step 
\item exception: N/A 
\end{itemize}

\noindent xdecrease():
\begin{itemize}
\item transition: if True 
\item output: Rotate the camera in the counter-clockwise horizontal direction for 1 step 
\item exception: N/A 
\end{itemize}

\noindent yincrease():
\begin{itemize}
\item transition: if True 
\item output: Rotate the camera in the vertical clockwise direction for 1 step
\item exception: N/A 
\end{itemize}

\noindent ydecrease():
\begin{itemize}
\item transition: if True 
\item output: Rotate the camera in the counter-clockwise horizontal direction for 1 step 
\item exception: N/A 
\end{itemize}

\noindent back\_to\_origin():
\begin{itemize}
\item transition: if True 
\item output: Rotate the camera to the original angle that has pre-set into the system
\item exception: N/A 
\end{itemize}




\subsubsection{Local Functions}

N/A

\newpage





\section{Appendix} \label{Appendix}

\wss{Extra information if required}


\newpage

\bibliographystyle {plainnat}
\bibliography {../../../refs/References}

\newpage

\end{document}
