\documentclass{article}

\usepackage{booktabs}
\usepackage{tabularx}
\usepackage{setspace}
\usepackage{fancyhdr}
\doublespacing
\pagestyle{fancy}
\fancyhead{} % clear all header fields
\fancyhead[C]{\textbf{Revision 0}}
\title{Development Plan\\\progname}

\author{\authname{Peng Cui&Yiwei He&Peihua Jin&Guangwei Tang&Erping Zhang}}

\date{09-24-2022}



\begin{document}
\maketitle
\section{Revisions}
\begin{table}[hp]
\caption{Revision History} \label{TblRevisionHistory}
\begin{tabularx}{\textwidth}{llX}
\toprule
\textbf{Date} & \textbf{Developer(s)} & \textbf{Change}\\
\midrule
Sept 24th 2022 & All & Revision 0\\
%Date2 & Name(s) & Description of changes\\

\bottomrule
\end{tabularx}
\end{table}

\newpage



%\wss{Put your introductory blurb here.}
\begin{large}
\section{Team Meeting Plan}
\hspace*{1cm} Team meeting will be hold once a week online on Wednesday after all team members' agreement. Meeting was scheduled from 10:30 to 11:30 which is a time slot no one has lecture going on. Period will be adjusted later on based on the real circumstances. Individual team member will be required to give a report on his work during previous week and the goal in next week. There will be no strict requirements but will be adjusted regarding schedule basis.
 
\section{Team Communication Plan}
\hspace*{1cm} Microsoft Teams will be the mean communication platform for team communication. Since work are divided into two parts, each small group will have private chat and the mean chat will be used for general discussion and integration later on. There will be no strict restriction. Communication will be conducted as needed basis.
\section{Team Member Roles}
\subsection {Hardware team(Erping Zhang, Guangwei Tang)}
\begin{itemize}
	\item Responsible for the computer vision human motion detection and tracking algorithm
	\item Servo control algorithm 
	\item Firmware/Platform decision and adaption
	\item CAD design using Invertor or SolidWork
	\item3D print and assembly
	\item Assist with software team
\end{itemize}
	
\subsection{Software team(Peng Cui, Peihua Jin, Yiwei He)}
\begin{itemize}
    \item Computer vision Object detection algorithm
	\item UI/UX design and implantation
	\item User experience design
	\item Test cases design and module units testing
	\item Optimize algorithm performance on the firmware
	\item Design the communication between software, camera and servo

\end{itemize}	
	









\section{Workflow Plan}

\begin{itemize}
	\item Design the new requirement and function of new module
	\item Design the test cases according to the requirements
	\item Pull the latest version from master repository
	\item Create a branch and only develop on one function or module
	\item Implement the module which is independent from other module or only use the finished/stable module
	\item Test the module according to test cases
	\item Check if there is any changes to the old finished module which the developing module calls during the development period(if yes, changes may be applied to the developing module)
	\item Merge the branch to the master repository after all members permission

\end{itemize}

\section{Proof of Concept Demonstration Plan}


\hspace{0.5cm}The main roadblock of the project is how to make the object detection algorithm accurate and distinguish the different objects using limit resources. Another technical risk is the detection of re-location of objects.

To overcome these risk, we will have a specific task-oriented logic in order to record the different status of objects such as re-location etc. Our plan is to use hand-tracking algorithm and frame-to-frame comparison method to activate the 'location-change' mode.

%%What is the main risk, or risks, for the success of your project?  What will you
%demonstrate during your proof of concept demonstration to convince yourself that
%you will be able to overcome this risk?%%

\section{Technology}

\begin{itemize}
\item Programming language: Python/C/C++
\item IDE: Visual Studio/PyCharm
\item Version Control Tool: Github/GitLab
\item Library: tKinter/OpenCV/numpy/PySerial
\item Unit testing framework: PyUnit
\item 3D modelling Tool: Inventor/SolidWork

\end{itemize}

\section{Coding Standard}
\hspace*{0.5cm} As python will be the primary language to be use. PEP 8 style will be applied during the development. As the major work will be divided into two parts, it is essential to keep the coding style consistent in integration stage. To be specific, code lay-out, naming convention and comments will be following the PEP 8 style in order to reduce the issues will be likely encountered in later development.
\section{Project Scheduling}
\hspace*{1cm} The whole framework is divided into hardware and software parts respectively. The role for members can refer to the team member role section. As the development is in the initial stage, the prototype design plan will likely be changed in future stages. The first milestone would be the decision making on hardware as well as the software language which will be done in the first few weeks. The second milestone would be the prototype algorithm of item identification. As the software part would play a major role in the design, 3 or 4 team member will concentrate on the algorithm design. The third milestone would be the completion of the software design and debugging. The final milestone would be the final integration of hardware and software.
%\wss{How will the project be scheduled?}
\end{large}
\end{document}
