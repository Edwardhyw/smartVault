\documentclass{article}

\usepackage{tabularx}
\usepackage{booktabs}

\title{Reflection Report on \progname}

\author{\authname}

\date{}

%% Comments

\usepackage{color}

\newif\ifcomments\commentstrue %displays comments
%\newif\ifcomments\commentsfalse %so that comments do not display

\ifcomments
\newcommand{\authornote}[3]{\textcolor{#1}{[#3 ---#2]}}
\newcommand{\todo}[1]{\textcolor{red}{[TODO: #1]}}
\else
\newcommand{\authornote}[3]{}
\newcommand{\todo}[1]{}
\fi

\newcommand{\wss}[1]{\authornote{blue}{SS}{#1}} 
\newcommand{\plt}[1]{\authornote{magenta}{TPLT}{#1}} %For explanation of the template
\newcommand{\an}[1]{\authornote{cyan}{Author}{#1}}

%% Common Parts

\newcommand{\progname}{Mechtronics Enigeering} % PUT YOUR PROGRAM NAME HERE
\newcommand{\authname}{Team 32, Wingman
\\ Edward He
\\ Erping Zhang
\\ Guangwei Tang
\\ Peng Cui
\\ Peihua Jin } % AUTHOR NAMES                  

\usepackage{hyperref}
    \hypersetup{colorlinks=true, linkcolor=blue, citecolor=blue, filecolor=blue,
                urlcolor=blue, unicode=false}
    \urlstyle{same}
                                


\begin{document}

\begin{table}[hp]
\caption{Revision History} \label{TblRevisionHistory}
\begin{tabularx}{\textwidth}{llX}
\toprule
\textbf{Date} & \textbf{Developer(s)} & \textbf{Change}\\
\midrule
2023-4-4 & All & Edit our team reflection for the final doc\\
\bottomrule
\end{tabularx}
\end{table}

\newpage

\maketitle

In this capstone reflection, we will discuss the development of a mechatronics system by our team of five Mechatronics students, focusing on the various skills required, personal growth, challenges faced, teamwork, and the project's future implications. Our project aimed to create an efficient, reliable, and functional mechatronics system combining hardware and software components, with a focus on real-time object detection and image processing.

\section{Project Overview}
Our mechatronics system is designed to efficiently detect objects in real-time and process images to provide valuable information for various applications, such as automation or monitoring. The project, spanning six months, involved a combination of hardware and software elements, including Python programming, image processing, user interface development, 3D modeling, and Arduino programming.
\section{Skill development and Task Delegation}
Our team identified several essential skills for the project and assigned tasks to individual members accordingly. Key skills and responsibilities included:
\begin{itemize}
    \item Python programming, image processing, and object detection: Edward, Peng, and Peihua
    \item User interface development: Edward, Erping, and Peihua
    \item Hardware-software communication: Erping and Guangwei
    \item 3D modeling: Guangwei
    \item Arduino programming: Erping and Guangwei
    \item Motor movement and control: entire team
    \item Team management and communication: entire team
\end{itemize}
\section{Teamwork and Collaboration}
As a team member, we actively participated in meetings, contributed to decision-making, and supported my teammates in their tasks. Our team's success can be attributed to the collaborative environment we fostered, which encouraged knowledge sharing, constructive feedback, and mutual support.
\section{Key Accomplishments}
Throughout the capstone project, our team achieved several significant accomplishments, which contributed to the project's success and demonstrated our ability to work effectively as a team. Some of the key accomplishments include:
\begin{enumerate}
    \item Development of a robust real-time object detection algorithm: Our team successfully researched, implemented, and optimized an efficient object detection algorithm in Python, which serves as the core of our mechatronics system.
    \item Creation of an intuitive user interface: We designed and developed a user-friendly web-based interface using JavaScript, allowing users to easily interact with our mechatronics system and access the processed information.
    \item Seamless hardware-software integration: Our team effectively integrated the software components with the hardware elements, ensuring smooth communication and overall system functionality.
    \item Efficient 3D modeling and printing: Guangwei successfully mastered 3D modeling skills, enabling us to design and 3D print a custom camera mount for the mechatronics system.
    \item Precise motor control using Arduino: Our team implemented precise motor control using Arduino programming, which allowed for accurate positioning and movement of the camera system.
    \item Strong team collaboration and communication: Throughout the project, our team maintained a high level of collaboration, openly sharing knowledge, providing constructive feedback, and working together to overcome challenges.
    \item Successful project management: By effectively delegating tasks, setting deadlines, and conducting regular progress meetings, our team was able to complete the project within the given timeframe and achieve the desired outcomes.
    \item Continuous learning and skill development: Each team member demonstrated a commitment to learning and mastering new skills, leading to personal growth and an increased ability to contribute to the project's success.
\end{enumerate}
These accomplishments not only showcase our team's technical expertise but also highlight our ability to work together, adapt to new challenges, and deliver a high-quality mechatronics system.
\section{Key Problem Areas}
Despite our accomplishments, our team faced several problem areas during the capstone project, which required us to adapt and find solutions to overcome these challenges. Some of the key problem areas include:
\begin{enumerate}
    \item Limited prior experience with specific technologies: Some team members had limited experience with certain technologies, such as Python programming, image processing, or Arduino programming. This initially slowed progress and required team members to invest additional time in learning and mastering these skills.
    \item Time constraints: Balancing the capstone project with coursework, internships, or other commitments posed challenges in allocating sufficient time to the project. Managing time effectively and prioritizing tasks became crucial to ensuring project completion within the given timeframe.
    \item Technical difficulties: Our team encountered technical issues, such as software bugs, hardware malfunctions, and integration challenges. Resolving these issues required troubleshooting, research, and collaboration among team members.
    \item Communication barriers: As with any team project, communication barriers occasionally arose, leading to potential misunderstandings or delays in progress. Regular team meetings, open communication channels, and active listening were essential in addressing these barriers.
    \item Resource limitations: Access to specific resources, such as hardware components, software tools, or expertise, was occasionally limited. Our team had to be resourceful, seeking alternative solutions or leveraging available resources to achieve project goals.
    \item Decision-making and consensus-building: Reaching consensus on key project decisions, such as selecting the appropriate object detection algorithm or designing the user interface, required effective communication and negotiation among team members to ensure alignment and commitment to the chosen path.
    \item Scope management: Maintaining the project's scope and avoiding unnecessary feature creep was a challenge. Our team had to be disciplined in defining project goals, prioritizing essential features, and evaluating the feasibility of implementing additional features within the project timeline.
\end{enumerate}
By identifying and addressing these problem areas, our team was able to learn valuable lessons, improve our project management and collaboration skills, and ultimately deliver a successful mechatronics system. Acknowledging and reflecting on these challenges will enable us to be better prepared for similar situations in future projects.
\section{What Would you Do Differently Next Time}
Reflecting on the capstone project experience, there are several aspects that we could approach differently in future projects to further improve our efficiency, collaboration, and overall outcomes. Some key considerations include:
\begin{enumerate}
    \item Establishing clear communication protocols: Implementing well-defined communication protocols, such as regular check-ins, progress reports, and designated communication channels, can help minimize misunderstandings and ensure that all team members are well-informed and aligned.
    \item Implementing a more structured project management approach: Adopting a formal project management methodology, such as Agile or Scrum, can help streamline the project's progress, promote adaptability, and enhance collaboration among team members.
    \item Encouraging continuous feedback and improvement: Foster a culture of continuous improvement by encouraging team members to share constructive feedback, discuss challenges openly, and collaboratively identify potential solutions or improvements.
    \item More frequent project milestones and reviews: Setting more frequent milestones and conducting regular project reviews can help track progress more effectively, identify potential issues early on, and make necessary adjustments in a timely manner.
    \item Allocating time for post-project evaluation: Schedule a dedicated period after the project's completion for the team to evaluate the outcomes, reflect on the process, and identify key lessons learned, which can be applied to future projects.
\end{enumerate}
By incorporating these lessons and adopting new strategies in future projects, our team can continue to grow and improve our ability to work together, overcome challenges, and deliver successful outcomes.
\section{Conclusion}
In conclusion, our capstone project provided a unique opportunity to apply the knowledge and skills gained throughout our academic journey. It challenged us to overcome obstacles, collaborate effectively, and grow both professionally and personally. The lessons learned from this project will undoubtedly prove valuable as we continue our careers in the field of mechatronics.
\end{document}