\documentclass[12pt]{article}

\usepackage{amsmath, mathtools}
\usepackage{amsfonts}
\usepackage{amssymb}
\usepackage{graphicx}
\usepackage{colortbl}
\usepackage{xr}
\usepackage{hyperref}
\usepackage{longtable}
\usepackage{xfrac}
\usepackage{tabularx}
\usepackage{float}
\usepackage{booktabs}
\usepackage{caption}
\usepackage{pdflscape}
\usepackage{afterpage}

\usepackage[round]{natbib}

\hypersetup{
    bookmarks=true,         % show bookmarks bar?
    colorlinks=true,       % false: boxed links; true: colored links
    linkcolor=red,          % color of internal links (change box color with linkbordercolor)
    citecolor=green,        % color of links to bibliography
    filecolor=magenta,      % color of file links
    urlcolor=cyan           % color of external links
}

%% Comments

\usepackage{color}

\newif\ifcomments\commentstrue %displays comments
%\newif\ifcomments\commentsfalse %so that comments do not display

\ifcomments
\newcommand{\authornote}[3]{\textcolor{#1}{[#3 ---#2]}}
\newcommand{\todo}[1]{\textcolor{red}{[TODO: #1]}}
\else
\newcommand{\authornote}[3]{}
\newcommand{\todo}[1]{}
\fi

\newcommand{\wss}[1]{\authornote{blue}{SS}{#1}} 
\newcommand{\plt}[1]{\authornote{magenta}{TPLT}{#1}} %For explanation of the template
\newcommand{\an}[1]{\authornote{cyan}{Author}{#1}}

%% Common Parts

\newcommand{\progname}{Mechtronics Enigeering} % PUT YOUR PROGRAM NAME HERE
\newcommand{\authname}{Team 32, Wingman
\\ Edward He
\\ Erping Zhang
\\ Guangwei Tang
\\ Peng Cui
\\ Peihua Jin } % AUTHOR NAMES                  

\usepackage{hyperref}
    \hypersetup{colorlinks=true, linkcolor=blue, citecolor=blue, filecolor=blue,
                urlcolor=blue, unicode=false}
    \urlstyle{same}
                                


\title{Software Requirements Specification\\\progname}

\author{\authname}

\date{}

\begin{document}

\maketitle

\newpage
\pagenumbering{roman}

\tableofcontents

\newpage

\begin{table}[hp]
\caption{Revision History} \label{TblRevisionHistory}
\begin{tabularx}{\textwidth}{llX}
\toprule
\textbf{Date} & \textbf{Developer(s)} & \textbf{Change}\\
\midrule
2022-10-04 & Edward He, Erping Zhang & Revision 0\\
& Guangwei Tang, Peng Cui & \\
& Peihua Jin & \\
\bottomrule
\end{tabularx}
\end{table}

\newpage

\listoftables
\listoffigures

\newpage

\pagenumbering{arabic}

This document describes the requirements for \progname. The template for the Software
Requirements Specification (SRS) is a subset of the Volere
template~\cite{RobertsonAndRobertson2012}. If you make further modifications
to the template, you should explicitly state what modifications were made.

\begin{table}

\end{table}

\section{Project Drivers}

\subsection{The Purpose of the Project}

\subsection{The Stakeholders}

\subsubsection{The Client}

\subsubsection{The Customers}

\subsubsection{Other Stakeholders}

\subsection{Mandated Constraints}

\subsection{Naming Conventions and Terminology}

\subsection{Relevant Facts and Assumptions}

User characteristics should go under assumptions.

\section{Functional Requirements}

\subsection{The Scope of the Work and the Product}

\subsubsection{The Context of the Work}

\subsubsection{Work Partitioning}

\subsubsection{Individual Product Use Cases}

\subsection{Functional Requirements}

\section{Non-functional Requirements}

\subsection{Look and Feel Requirements}

\subsection{Usability and Humanity Requirements}

\subsection{Performance Requirements}

\subsection{Operational and Environmental Requirements}

\subsection{Maintainability and Support Requirements}

\subsection{Security Requirements}

\subsection{Cultural Requirements}

\subsection{Legal Requirements}

\subsection{Health and Safety Requirements}

This section is not in the original Volere template, but health and safety are
issues that should be considered for every engineering project.

\section{Project Issues}

\subsection{Open Issues}

\begin{itemize}
    \item Limit of 180 rotation degree of servo motor
    \item Accuracy of object detection 
    \item How to distinguish two objects with very limited resources
    \item How to recognize the same object in different angles
    \item How to guarantee the stability of the serial communication
\end{itemize}


\subsection{Off-the-Shelf Solutions}
\begin{itemize}

    \item Huskylens - is a AI camera which can learn new objects and recognize them. It has the machine learning technology enables projects to interact with people and environments which allows many kinds of system control.
    \item NVIDIA Jetson TX2 - is an embedded AI computer device. It has 8GB memory and 59.7GB/s of memory bandwidth which provides a high quality AI performance to build efficient AI models including computer vision.
\end{itemize}


\subsection{New Problems}
    \subsubsection{Effects on the Current Environment}
       \hspace{0.5cm} Any changes to the exist database may cause the data related to each object missing
    \subsubsection{Effect on the Installed Systems}
        \hspace{0.5cm} Changes to the motorized camera mount will affect the algorithm or logic of the controller board
    \subsubsection{Potential User Problems}
        \hspace{0.5cm} Changes to the user interface may change the way that user used to search the object
    
    \subsubsection{Limitations in the Anticipated Implementation Environment}
        \hspace{0.5cm}NA
    \subsubsection{Follow-Up Problems}
        \hspace{0.5cm} The changes in computer vision algorithm may cause the whole system malfunction

\subsection{Tasks}
    \subsubsection{Project Planning}
    NA
    
    \subsubsection{Planning of the Development Phases}
    NA
\subsection{Migration to the New Product}
    \subsubsection{Requirements for Migration to the New Product}
        \begin{itemize}
            \item All the objects data should be stored 
            \item Motorized camera mount should be calibrated before using
            \item The spec of camera should be kept in same as possible
        \end{itemize}
    \subsubsection{Data that Has to be Modified or translated for the new system}
        \begin{itemize}
            \item Objects data need to be transfer ed into the new system
        \end{itemize}

\subsection{Risks}
\begin{itemize}
    \item Connection lost between the board and camera during the motor movement.
    \item Inappropriate distance measure and control between the object and the motor which furthermore cause damage or stuck.
    \item Physical damage from collision with objects .
    \item Physical damage from wire twisting during rotation.
    \item Unexpected movement caused by the delay of data transfer
\end{itemize}
\subsection{Costs}
\begin{large}
\begin{center}
    \begin{tabular}{|| c || c ||}
    \hline
    Product & Price\\
    \hline\hline
    USB Camera & \$30\\
    \hline
    PTZ Mount & \$25\\
    \hline
    Arduino & \$30\\
    \hline
    Motors & \$15\\
    \hline
    Total & \$100\\
    \hline
    \end{tabular}
\end{center}
\end{large}
\subsection{User Documentation and Training}
    \subsubsection{User Documentation Requirements}
    \begin{itemize}
        \item User manuals
        \item Installation manuals
        \item Technical specifications to accompany the product
    \end{itemize}
    
    \subsubsection{Training Requirements}
    NA
\subsection{Waiting Room}
\begin{itemize}
    \item Expanding chassis's activity area including rotation angle and planar movement.
    \item Developing new algorithm regarding data transfer to enable faster real-time reaction.
    \item Adding alarm in case that object not found in the assigned area as an application of storage security.
\end{itemize}
\subsection{Ideas for Solutions}
\begin{itemize}
    \item Use DC motor with decoder to implement the unlimited rotation degree of camera mount
    \item Use Field Oriented Control algorithm to implement the unlimited rotation degree of camera mount
    \item Use frame-to-frame compassion to detect the relocation of objects
    \item Predict the possible location of objects by tracking the users path in the room
    \item Beeper alert when the camera view is not cleared
\end{itemize}


\subsection{Off-the-Shelf Solutions}

\subsection{New Problems}

\subsection{Tasks}

\subsection{Migration to the New Product}

\subsection{Risks}

\subsection{Costs}

\subsection{User Documentation and Training}

\subsection{Waiting Room}

\subsection{Ideas for Solutions}


\newpage

\bibliographystyle {plainnat}
\bibliography {../../refs/References}

\newpage

\section{Appendix}

This section has been added to the Volere template.  This is where you can place
additional information.

\subsection{Symbolic Parameters}

The definition of the requirements will likely call for SYMBOLIC\_CONSTANTS.
Their values are defined in this section for easy maintenance.

\subsection{Reflection}

The information in this section will be used to evaluate the team members on the
graduate attribute of Lifelong Learning.  Please answer the following questions:

\begin{enumerate}
  \item What knowledge and skills will the team collectively need to acquire to
  successfully complete this capstone project?  Examples of possible knowledge
  to acquire include domain specific knowledge from the domain of your
  application, or software engineering knowledge, mechatronics knowledge or
  computer science knowledge.  Skills may be related to technology, or writing,
  or presentation, or team management, etc.  You should look to identify at
  least one item for each team member.
  \item For each of the knowledge areas and skills identified in the previous
  question, what are at least two approaches to acquiring the knowledge or
  mastering the skill?  Of the identified approaches, which will each team
  member pursue, and why did they make this choice?
\end{enumerate}

\end{document}