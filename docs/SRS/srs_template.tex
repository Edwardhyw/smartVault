\documentclass[12pt]{article}

\usepackage{amsmath, mathtools}
\usepackage{amsfonts}
\usepackage{amssymb}
\usepackage{graphicx}
\usepackage{colortbl}
\usepackage{xr}
\usepackage{hyperref}
\usepackage{longtable}
\usepackage{xfrac}
\usepackage{tabularx}
\usepackage{float}
\usepackage{booktabs}
\usepackage{caption}
\usepackage{pdflscape}
\usepackage{afterpage}

\usepackage[round]{natbib}

\hypersetup{
    bookmarks=true,         % show bookmarks bar?
    colorlinks=true,       % false: boxed links; true: colored links
    linkcolor=red,          % color of internal links (change box color with linkbordercolor)
    citecolor=green,        % color of links to bibliography
    filecolor=magenta,      % color of file links
    urlcolor=cyan           % color of external links
}

\input{../Comments}
%% Common Parts

\newcommand{\progname}{Mechtronics Enigeering} % PUT YOUR PROGRAM NAME HERE
\newcommand{\authname}{Team 32, Wingman
\\ Edward He
\\ Erping Zhang
\\ Guangwei Tang
\\ Peng Cui
\\ Peihua Jin } % AUTHOR NAMES                  

\usepackage{hyperref}
    \hypersetup{colorlinks=true, linkcolor=blue, citecolor=blue, filecolor=blue,
                urlcolor=blue, unicode=false}
    \urlstyle{same}
                                


\title{Software Requirements Specification\\\progname}

\author{\authname}

\date{}

\begin{document}

\maketitle

\newpage
\pagenumbering{roman}

\tableofcontents

\newpage

\begin{table}[hp]
\caption{Revision History} \label{TblRevisionHistory}
\begin{tabularx}{\textwidth}{llX}
\toprule
\textbf{Date} & \textbf{Developer(s)} & \textbf{Change}\\
\midrule
2022-10-04 & Edward He, Erping Zhang & Revision 0\\
& Guangwei Tang, Peng Cui & \\
& Peihua Jin & \\
\bottomrule
\end{tabularx}
\end{table}

\newpage

\listoftables
\listoffigures

\newpage

\pagenumbering{arabic}

This document describes the requirements for \progname. The template for the Software
Requirements Specification (SRS) is a subset of the Volere
template~\cite{RobertsonAndRobertson2012}. If you make further modifications
to the template, you should explicitly state what modifications were made.

\begin{table}

\end{table}

\section{Project Drivers}

\subsection{The Purpose of the Project}
The purpose of the project is to build a mechatronics system called "SmartVault" that is able to assist user in finding their belongings in a given area. 
\subsection{The Purpose of the Document}
This document is intended to provide detailed set of requirements project SmartVault. The documentation will cover the functionality of the system and the requirements that the system is expected to fulfill. In addition to the function requirements of software system, non-functional requirements will also be included in this document. Any additional useful information for building the system is also covered and written in details. This document is used as a reference and guideline for the development of the system and is to ensure that the built system is fulfilling the necessary requirements and meeting the desired goals.
\subsection{The Stakeholders}

\subsubsection{The Client}
\begin{itemize}
    \item Dr. Spencer Smith, professor from McMaster university, Computing and Software department. 
\end{itemize}

\subsubsection{The Customers}
\begin{itemize}
    \item people who often can't find their belongings due to poor organization
   	\item people who has bad memory  
\end{itemize}
\subsubsection{Other Stakeholders}
N/A
\section{Project Constraints}
\subsection{Mandated Constraints}

\subsection{Naming Conventions and Terminology}

\subsection{Relevant Facts and Assumptions}

\subsubsection{User Characteristics}
The users of SmartVault are people who have trouble finding their belongings in their daily lives. It is preferred that the user to have general knowledge of what is presented in the designated area. Users should be familiar and comfortable with technology. At minimum, the users should be able to read, type and navigate on a webpage. 
\subsubsection{Reader Characteristics}
The intended readers for this document are the developers responsible for developing SmartVault. Dr.Smith, a professor at McMaster University and his teaching team will also be the intended reader of this document. The document is written in a technical manner and requires the reader to possess both software and hardware knowledge. The reader should be familiar with image processing and the ability to understand the basic logic of object detection. The reader should also understand the software design cycle and the steps taken to develop a complete software program. In order to fully understand the document, it is suggested that the reader has the ability to read and comprehend the communication and control system between a software program and hardware device. 
\section{Functional Requirements}

\subsection{The Scope of the Work and the Product}

\subsubsection{The Context of the Work}

\subsubsection{Work Partitioning}

\subsubsection{Individual Product Use Cases}

\subsection{Functional Requirements}

\section{Non-functional Requirements}

\subsection{Look and Feel Requirements}

\subsection{Usability and Humanity Requirements}

\subsection{Performance Requirements}

\subsection{Operational and Environmental Requirements}

\subsection{Maintainability and Support Requirements}

\subsection{Security Requirements}

\subsection{Cultural Requirements}

\subsection{Legal Requirements}

\subsection{Health and Safety Requirements}

This section is not in the original Volere template, but health and safety are
issues that should be considered for every engineering project.

\section{Project Issues}

\subsection{Open Issues}

\subsection{Off-the-Shelf Solutions}

\subsection{New Problems}

\subsection{Tasks}

\subsection{Migration to the New Product}

\subsection{Risks}

\subsection{Costs}

\subsection{User Documentation and Training}

\subsection{Waiting Room}

\subsection{Ideas for Solutions}

\newpage

\bibliographystyle {plainnat}
\bibliography {../../refs/References}

\newpage

\section{Appendix}

This section has been added to the Volere template.  This is where you can place
additional information.

\subsection{Symbolic Parameters}

The definition of the requirements will likely call for SYMBOLIC\_CONSTANTS.
Their values are defined in this section for easy maintenance.

\subsection{Reflection}

Our team consists of 5 mechatronics student and is developing a mechatronics system where both hardware and software components are essential for success. There are may different skills and knowledge that ties into the success of the project. Many of these required skills might new and unfamiliar to some of the group members. However, as this is a project over a span of 6 months, there are plenty of time and opportunities to pick up and master some new skills. The learning of such skills will be delegated to different team members and mastered by them before delivering and teaching others member about it.  

The project is heavily relying on a functional, reliable, and efficient software system that involve live time object detection and image processing. The coding will be done in Python, so each member needs to have adequate skill of programming in Python and able to develop and test a python program. Knowledge of image processing and object detection also should be acquired.  In order to master the skills for programming in python and image processing, team members should practice and conduct research on different object detection method. It is required for the assigned team members to compare and learn from different method and making sure that our chosen method would yield the best results given our requirements. This part of the learning is assigned to Edward, Peng and Jinhua. Edward has gained relevant knowledge through courses he took and projects he did. Peng and Jinhua have great enthusiasm for this topic and would like to conduct research on it. 

Another software component is the user interface. Members of the team should be able to build and maintain a working webpage using JavaScript. This skill is obtained by Edward through his co-op and other experiences. Erping and Jinhua also are assigned to perfect their abilities in such area. The path to success for this is to read relevant documents and tutorials, as well as practicing with small projects.
In order to communicate between the software and hardware, related programming skills are needed. Erping as the lead developer for this topic has multiple experiences and with the help of Guangwei who has exceptional skills in this field, they will be able to excel the required skills and guide other members of the team.  

On the hardware side, Guangwei will be the lead and responsible for the learning and teaching of the related skills. 3D modeling is needed for the building and 3D printing the camera mount. This skill is assigned to Guangwei and can be mastered through practicing and watching tutorials. Arduino board is the chosen board for handling the control of the motors and repositioning of the camera. Sufficient knowledge and skills to program Arduino is needed. This task is assigned to Erping and Guangwei. They both have experiences in this field and will be able to teach the rest of the team. Knowledge associated with handling motor movement is also needed and can be mastered by reviewing course notes and refreshing past knowledge. 

Aside from the hardware and software system, there are also skills associated with team management and team communication. In order to maintain a desired work plan and team coordination, it is important for all members of the team to possess the skills to work under a team environment. Members need to be comfortable and familiar with github and in order to excel, one must practice different github actions and commends. One also can improve on this by watching tutorial videos. Working in a team environment, it is important to have good communication skills and making sure people are on the same page. This can be mastered by actively engage in team meetings and ask for help when needed.  

To conclude, for a professional project like the capstone, there are many aspects to it and many skills are required. It is important to delegate different learning tasks to individual and be able to teach the new skillset to the rest of the team. The following are the skills mentioned in previous paragraphs and is the overview of the assignment of the learning process:


\begin{enumerate}
  \item communication and team management (All)
  \item Github (All)
  \item Python, image processing and object detection (Edward, Jinhua, Peng)
  \item Arduino and control (Erping and Guangwei)
  \item Motor control (Erping and Guangwei)
  \item Javascript (Edward, Peng, Jinhua)
  \item 3D modeling and 3D printing (Erping and Guangwei)
  \item Writing and presenting reports (All)
\end{enumerate}

\end{document}