\documentclass[12pt, titlepage]{article}

\usepackage{booktabs}
\usepackage{tabularx}
\usepackage{hyperref}
\hypersetup{
    colorlinks,
    citecolor=blue,
    filecolor=black,
    linkcolor=red,
    urlcolor=blue
}
\usepackage[round]{natbib}

%% Comments

\usepackage{color}

\newif\ifcomments\commentstrue %displays comments
%\newif\ifcomments\commentsfalse %so that comments do not display

\ifcomments
\newcommand{\authornote}[3]{\textcolor{#1}{[#3 ---#2]}}
\newcommand{\todo}[1]{\textcolor{red}{[TODO: #1]}}
\else
\newcommand{\authornote}[3]{}
\newcommand{\todo}[1]{}
\fi

\newcommand{\wss}[1]{\authornote{blue}{SS}{#1}} 
\newcommand{\plt}[1]{\authornote{magenta}{TPLT}{#1}} %For explanation of the template
\newcommand{\an}[1]{\authornote{cyan}{Author}{#1}}

%% Common Parts

\newcommand{\progname}{Mechtronics Enigeering} % PUT YOUR PROGRAM NAME HERE
\newcommand{\authname}{Team 32, Wingman
\\ Edward He
\\ Erping Zhang
\\ Guangwei Tang
\\ Peng Cui
\\ Peihua Jin } % AUTHOR NAMES                  

\usepackage{hyperref}
    \hypersetup{colorlinks=true, linkcolor=blue, citecolor=blue, filecolor=blue,
                urlcolor=blue, unicode=false}
    \urlstyle{same}
                                


\begin{document}

\title{Project Title: System Verification and Validation Plan for \progname{}} 
\author{\authname}
\date{\today}
	
\maketitle

\pagenumbering{roman}

\section{Revision History}

\begin{tabularx}{\textwidth}{p{3cm}p{2cm}X}
\toprule {\bf Date} & {\bf Version} & {\bf Notes}\\
\midrule
Date 1 & 1.0 & Notes\\
Date 2 & 1.1 & Notes\\
\bottomrule
\end{tabularx}

\newpage

\tableofcontents

\listoftables
\wss{Remove this section if it isn't needed}

\listoffigures
\wss{Remove this section if it isn't needed}

\newpage

\section{Symbols, Abbreviations and Acronyms}

\renewcommand{\arraystretch}{1.2}
\begin{tabular}{l l} 
  \toprule		
  \textbf{symbol} & \textbf{description}\\
  \midrule 
  T & Test\\
  \bottomrule
\end{tabular}\\

\wss{symbols, abbreviations or acronyms -- you can simply reference the SRS
  \citep{SRS} tables, if appropriate}

\newpage

\pagenumbering{arabic}

This document ... \wss{provide an introductory blurb and roadmap of the
  Verification and Validation plan}

\section{General Information}

\subsection{Summary}

\wss{Say what software is being tested.  Give its name and a brief overview of
  its general functions.}

\subsection{Objectives}

\wss{State what is intended to be accomplished.  The objective will be around
  the qualities that are most important for your project.  You might have
  something like: ``build confidence in the software correctness,''
  ``demonstrate adequate usability.'' etc.  You won't list all of the qualities,
  just those that are most important.}

\subsection{Relevant Documentation}

\wss{Reference relevant documentation.  This will definitely include your SRS
  and your other project documents (MG, MIS, etc).  You can include these even
  before they are written, since by the time the project is done, they will be
  written.}

\citet{SRS}

\section{Plan}
In this section, it provides an overview and roadmap for the document. This section outlines the members of the verification and validation team for the project SmartVault and their assigned roles. This section describes the methods to verify and validate each component of the project, which includes SRS, Design, implementation, testing tools and the software for SmartVault.

\subsection{Verification and Validation Team}

\begin{table}[h]
\begin{center}
\begin{tabular}{|l | l|}
\hline
  \textbf{Member} & \textbf{Role}\\
  \hline
  Edward He & Static Verification Lead; Back-end Testing\\
  \hline
  Erping Zhang & Design Verification Lead; Front-end Testing \\
  \hline
  Guangwei Tang & SRS Verification Lead; Hardware Testing\\
  \hline
  Peng Cui & Dynamic Verification Lead; Code Inspection\\
  \hline
  Peihua Jin & Automated Testing Lead;Code Inspection\\
  \hline
  Spencer Smith & Review of documents and system\\
  \hline
  Peer Groups & Review of documents and system\\
  \hline
\end{tabular}
\end{center}
\caption{Verification and Validation Team Members and Roles}            

\end{table}
\break
\subsection{SRS Verification Plan}

In order to verify the SRS document, the SRS checklist will be used to make sure that the document included all components on the checklist and are complete. The SRS document will also be reviewed by other groups and based on their feedbacks, changes will be made accordingly. Professor Spencer Smith and the teaching team will also review the document and provide feedbakcs. Requirements Listed in the SRS document will be verified and validated by the methods listed in the following sections.



\subsection{Design Verification Plan}

To verify the design of the system, MG and MIS checklist will be utilized. The team has decided and implemented a procedure for each individual design made by different team developer. Any design decision is required to be reviewed and verified by at least 3 other members. The design of SmartVault will also be reviewed by peers and instructor. 

\subsection{Implementation Verification Plan}

The implementation of the SmartVault will be done with both dynamic and static verification. For dynamic verification, both non-functional and functional requirements implementation will be tested. The test will consist both system test and unit test to ensure the written code comply with the requirements and functionalities. More details for the dynamic verification refer to section 5 and 6 of this document. For the static verification, it will be done with code inspection and coding standards. The team will utilize  Git and Github to inspect and verify coding. The merge of a new functionality or modification of current implementation has to be reviewed by 2 others member prior to merging to the main branch. Code inspections/walkthrough will be done weekly to ensure it follows the coding standards. SmartVault will be primarily written in Python and team members are required to follow PEP8 standard by using linter Pylint.
\subsection{Automated Testing and Verification Tools}

\wss{What tools are you using for automated testing.  Likely a unit testing
  framework and maybe a profiling tool, like ValGrind.  Other possible tools
  include a static analyzer, make, continuous integration tools, test coverage
  tools, etc.  Explain your plans for summarizing code coverage metrics.
  Linters are another important class of tools.  For the programming language
  you select, you should look at the available linters.  There may also be tools
  that verify that coding standards have been respected, like flake9 for
  Python.}

\wss{The details of this section will likely evolve as you get closer to the
  implementation.}

\subsection{Software Validation Plan}

Software validation will be mostly conducted by team members. External validation is set to be conducted with consumers(stakeholders) of the product where they can try out the system and report if there is anything they would like to change.

\section{System Test Description}
	
\subsection{Tests for Functional Requirements}

\wss{Subsets of the tests may be in related, so this section is divided into
  different areas.  If there are no identifiable subsets for the tests, this
  level of document structure can be removed.}

\wss{Include a blurb here to explain why the subsections below
  cover the requirements.  References to the SRS would be good.}

\subsubsection{Area of Testing1}

\wss{It would be nice to have a blurb here to explain why the subsections below
  cover the requirements.  References to the SRS would be good.  If a section
  covers tests for input constraints, you should reference the data constraints
  table in the SRS.}
		
\paragraph{Title for Test}

\begin{enumerate}

\item{test-id1\\}

Control: Manual versus Automatic
					
Initial State: 
					
Input: 
					
Output: \wss{The expected result for the given inputs}

Test Case Derivation: \wss{Justify the expected value given in the Output field}
					
How test will be performed: 
					
\item{test-id2\\}

Control: Manual versus Automatic
					
Initial State: 
					
Input: 
					
Output: \wss{The expected result for the given inputs}

Test Case Derivation: \wss{Justify the expected value given in the Output field}

How test will be performed: 

\end{enumerate}

\subsubsection{Area of Testing2}

...

\subsection{Tests for Nonfunctional Requirements}

\wss{The nonfunctional requirements for accuracy will likely just reference the
  appropriate functional tests from above.  The test cases should mention
  reporting the relative error for these tests.}

\wss{Tests related to usability could include conducting a usability test and
  survey.}

\subsubsection{Area of Testing1}
		
\paragraph{Title for Test}

\begin{enumerate}

\item{test-id1\\}

Type: 
					
Initial State: 
					
Input/Condition: 
					
Output/Result: 
					
How test will be performed: 
					
\item{test-id2\\}

Type: Functional, Dynamic, Manual, Static etc.
					
Initial State: 
					
Input: 
					
Output: 
					
How test will be performed: 

\end{enumerate}

\subsubsection{Area of Testing2}

...

\subsection{Traceability Between Test Cases and Requirements}

\wss{Provide a table that shows which test cases are supporting which
  requirements.}
\section{Proof of Concept Testing}
Before any serious development of the game starts, a proof-of-concept test would be carried out to show the undertaking is feasible. The remaining of this section describes the proof-of-concept test in detail.

\subsection{Significant Risks}
The successful completion of the project depends on overcoming the following significant risks:
\begin{enumerate}
\item{}
In order to satisfy the functionality of human detection, the camera must be able to rotate and follow the movement of the user. The risk is to make the user located in the center of the screen all the time. Otherwise, the observation may behave poorly and unclearly if the user is out of the screen or located at the edge of the screen.			
\item{}
SmartVault is used to keep the items of users or to find out when and what gets lost or missed. The most important part of the project is to identify the differences through the compare between two images.
\end{enumerate}
\subsection{Demonstration Plan}
The prototype will try to test human detection firstly. Three cases will be applied for the proof-of-concept testing.
\begin{enumerate}
\item{test 1\\}
Type: Functional, automatic\\
Initial State: User stays in the center of the screen\\
Input: Null\\
Output: The camera does not rotate, and the user is identified with a yellow rectangle shown in the screen.\\			
\item{test 2\\}
Type: Functional, dynamic, automatic\\
Initial State: User stays at the edge of the screen\\
Input: Null\\
Output: The user is identified with a yellow rectangle shown in the screen, then the camera begins to rotate until the user is in the center of the screen.\\
\item{test 3\\}
Type: Functional, automatic\\
Initial State: User is out of the screen\\
Input: Null\\
Output: The camera does not rotate, and the user cannot be identified on the screen.\\
\item{test 4\\}
Type: Functional, dynamic, automatic\\
Initial State: User is in the screen\\
Input: User randomly walk around the room\\
Output: The camera rotates and follows the user, and the user is identified with a yellow rectangle shown in the screen. In addition, the location of the rectangle also moves depending on the movement of the user.\\
\end{enumerate}
The prototype will try to test item identification secondly. Three cases will be applied for the proof-of-concept testing.
\begin{enumerate}
\item{test 1\\}
Type: Functional, static, automatic\\
Initial State: Null\\
Input: Two identical images with no difference.\\
Output: Nothing to do with SmartVault.\\ 
\item{test 2\\}
Type: Functional, static, automatic\\
Initial State: Null\\
Input: Two images with the same item, but the location of the item is changed.\\
Output: The difference of the location is identified. SmartVault will record these change and upload to the local file.\\ 
\item{test 3\\}
Type: Functional, static, automatic\\
Initial State: Null\\
Input: Two images with different items, but the locations of the items are the same.\\
Output: The difference of the items is identified. SmartVault will record these change and upload to the local file.\\ 
\item{test 4\\}
Type: Functional, static, automatic\\
Initial State: Null\\
Input: A image with 6 small items.\\
Output: SmartVault will identify each item with a yellow rectangle around it. In other words, there should be 6 rectangles in total.\\ 
\item{test 5\\}
Type: Functional, dynamic, automatic\\
Initial State: Null\\
Input: A image with 1 small item and it will keep moving during the test.\\
Output: SmartVault will identify this item with a yellow rectangle around it. In addition, the rectangle will move depending on the movement of the object\\ 
\end{enumerate}

\section{Unit Test Description}

\subsection{Unit Testing of Internal Functions}
In order to build unit tests for the internal functions of the program, certain modules which can return values could be tested. This will involve choosing proper method and giving them input values. Through comparing what they are expected to output and the actual outputs, a series of unit tests can be created. These unit tests will choose both proper inputs and those could generate exceptions. The acceptable range for the unit test could be determined through testing the output with exceptions. There is no need for stubs or drivers to be applied for the testing. Since the tests have been already done by the individual classes. Some coverage metrics will be shown to measure how much of the code had been covered for testing purposes. The goal is to reach as much as possible to ensure that all functions related to various modules get tested adequately. The expectation of the coverage percentage of unit tests is set to 80 percent.
\subsection{Unit Testing of Output Files}
In order to create unit tests for the output files, the quality and accuracy of the output images will be considered for testing. This will involve creating certain situation and compare the actual results with the expected ones. For the saved images in the output files, they will be rated depending on how similar they are with the expected images. Only those output images being rated with a score larger than 95 percent will be considered as qualified. On the other hand, if the score does not meet 95 percent, the test case fails and the developer team should analysis the reason behind it and redo the test case. In addition, some extreme cases will be applied to the unit testing process, to measure the maximum and minimum acceptable range for the unit tests.

				
\bibliographystyle{plainnat}

\bibliography{../../refs/References}

\newpage

\section{Appendix}

This is where you can place additional information.

\subsection{Symbolic Parameters}

The definition of the test cases will call for SYMBOLIC\_CONSTANTS.
Their values are defined in this section for easy maintenance.

\subsection{Usability Survey Questions?}

\wss{This is a section that would be appropriate for some projects.}

\newpage{}
\section*{Appendix --- Reflection}

The information in this section will be used to evaluate the team members on the
graduate attribute of Lifelong Learning.  Please answer the following questions:

\begin{enumerate}
  \item 
  \item 
\end{enumerate}

\end{document}
