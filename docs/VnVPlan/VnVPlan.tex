\documentclass[12pt, titlepage]{article}

\usepackage{booktabs}
\usepackage{tabularx}
\usepackage{hyperref}
\hypersetup{
    colorlinks,
    citecolor=blue,
    filecolor=black,
    linkcolor=red,
    urlcolor=blue
}
\usepackage[round]{natbib}

%% Comments

\usepackage{color}

\newif\ifcomments\commentstrue %displays comments
%\newif\ifcomments\commentsfalse %so that comments do not display

\ifcomments
\newcommand{\authornote}[3]{\textcolor{#1}{[#3 ---#2]}}
\newcommand{\todo}[1]{\textcolor{red}{[TODO: #1]}}
\else
\newcommand{\authornote}[3]{}
\newcommand{\todo}[1]{}
\fi

\newcommand{\wss}[1]{\authornote{blue}{SS}{#1}} 
\newcommand{\plt}[1]{\authornote{magenta}{TPLT}{#1}} %For explanation of the template
\newcommand{\an}[1]{\authornote{cyan}{Author}{#1}}

%% Common Parts

\newcommand{\progname}{Mechtronics Enigeering} % PUT YOUR PROGRAM NAME HERE
\newcommand{\authname}{Team 32, Wingman
\\ Edward He
\\ Erping Zhang
\\ Guangwei Tang
\\ Peng Cui
\\ Peihua Jin } % AUTHOR NAMES                  

\usepackage{hyperref}
    \hypersetup{colorlinks=true, linkcolor=blue, citecolor=blue, filecolor=blue,
                urlcolor=blue, unicode=false}
    \urlstyle{same}
                                


\begin{document}

\title{Project Title: System Verification and Validation Plan for \progname{}} 
\author{\authname}
\date{\today}
	
\maketitle

\pagenumbering{roman}

\section{Revision History}

\begin{tabularx}{\textwidth}{p{3cm}p{2cm}X}
\toprule {\bf Date} & {\bf Version} & {\bf Notes}\\
\midrule
Date 1 & 1.0 & Notes\\
Date 2 & 1.1 & Notes\\
\bottomrule
\end{tabularx}

\newpage

\tableofcontents

\listoftables
\wss{Remove this section if it isn't needed}

\listoffigures
\wss{Remove this section if it isn't needed}

\newpage

\section{Symbols, Abbreviations and Acronyms}

\renewcommand{\arraystretch}{1.2}
\begin{tabular}{l l} 
  \toprule		
  \textbf{symbol} & \textbf{description}\\
  \midrule 
  T & Test\\
  \bottomrule
\end{tabular}\\

\wss{symbols, abbreviations or acronyms -- you can simply reference the SRS
  \citep{SRS} tables, if appropriate}

\newpage

\pagenumbering{arabic}

This document ... \wss{provide an introductory blurb and roadmap of the
  Verification and Validation plan}

\section{General Information}

\subsection{Summary}

\wss{The SmartVault project is a Mechatronics system that is able to assist user identify the location of assigned objects in a given area.}

\subsection{Objectives}

\wss{This document is intended to develop a systematic plan for testing the functionality of the system.It meant to show the system has met the requirements in both software and hardware aspects mentioned in requirements document.In this document, system test and unit test will be conducted to check the functionality of the system. Functional and non-functional requirements will be separated for testing for both section. By the end of testing process, it can be shown that the system is working properly and available for usage.}

\subsection{Relevant Documentation}

\wss{In this document, Requirements Document will be referenced where the tests will be conducted based on the system requirements.}

\citet{SRS}

\section{Plan}

\wss{Introduce this section.   You can provide a roadmap of the sections to
  come.}

\subsection{Verification and Validation Team}

\wss{You, your classmates and the course instructor.  Maybe your supervisor.
  You shoud do more than list names.  You should say what each person's role is
  for the project.  A table is a good way to summarize this information.}

\subsection{SRS Verification Plan}

\wss{List any approaches you intend to use for SRS verification.  This may just
  be ad hoc feedback from reviewers, like your classmates, or you may have
  something more rigorous/systematic in mind..}

\wss{Remember you have an SRS checklist}

\subsection{Design Verification Plan}

\wss{Plans for design verification}

\wss{The review will include reviews by your classmates}

\wss{Remember you have MG and MIS checklists}

\subsection{Implementation Verification Plan}

\wss{You should at least point to the tests listed in this document and the unit
  testing plan.}

\wss{In this section you would also give any details of any plans for static verification of
  the implementation.  Potential techniques include code walkthroughs, code
  inspection, static analyzers, etc.}

\subsection{Automated Testing and Verification Tools}

\wss{What tools are you using for automated testing.  Likely a unit testing
  framework and maybe a profiling tool, like ValGrind.  Other possible tools
  include a static analyzer, make, continuous integration tools, test coverage
  tools, etc.  Explain your plans for summarizing code coverage metrics.
  Linters are another important class of tools.  For the programming language
  you select, you should look at the available linters.  There may also be tools
  that verify that coding standards have been respected, like flake9 for
  Python.}

\wss{The details of this section will likely evolve as you get closer to the
  implementation.}

\subsection{Software Validation Plan}

\wss{If there is any external data that can be used for validation, you should
  point to it here.  If there are no plans for validation, you should state that
  here.}

\section{System Test Description}
	
\subsection{Tests for Functional Requirements}

\subsubsection{Area of Testing1}
		
\paragraph{Manual Testing}

\begin{enumerate}

\item{IPR1-1\\}

Control: Manual
					
Initial State: The system is turned on, the working environment should be empty
					
Input: Images of the working environment and a human show up in the environment
					
Output: \wss{Coordinate of the detected human body}

Test Case Derivation: \wss{The output coordinate of the detected human body should located on the human body in the image}
					
How test will be performed: Turn on the system. Start running the system without the human showed up in the environment. Then the human should move into the environment and the system will put a square shape on the human body in the image.


\item{IPR2-1\\}

Control: Manual
					
Initial State: The system is turned on, the working environment should be empty
					
Input: Images of the working environment and a hand show up in the environment
					
Output: \wss{Coordinate of the detected hand and each joints}

Test Case Derivation: \wss{The output coordinate of the detected hand and joints should located on the joints in the image}
					
How test will be performed: Turn on the system. Start running the system without the hand showed up in the environment. Then the hand should move into the environment and the system will put dots on each joints of the hand in the image.

\item{IPR3-1\\}

Control: Manual
					
Initial State: The system is turned on, the working environment should be empty
					
Input: Images of the working environment and few objects in the environment
					
Output: \wss{Coordinate of the detected objects}

Test Case Derivation: \wss{The output coordinate of the detected objects should located on the object in the image}
					
How test will be performed: Turn on the system. Start running the system without the objects showed up in the environment. Then the objects should move into the environment and the system will put square on each object in the image.


\item{IPR4-1\\}

Control: Manual
					
Initial State: The system is turned on, the working environment should include at least one item.
					
Input: Images of the working environment and object in the environment. Different Images should have different location of the object in the environment.
					
Output: \wss{The re-location mode should be activated}

Test Case Derivation: \wss{The re-location mode should be activated when the object location change}
					
How test will be performed: Turn on the system. Start running the system with the objects showed up in the environment. Then change the location of object in the environment, the system should activated the re-location mode.


\item{IPR5-1\\}

Control: Manual
					
Initial State: The system is turned on, the working environment should include at least two items.
					
Input: Images of the working environment and at least two different objects in the environment. 
					
Output: \wss{The system should record two different objects into the database}

Test Case Derivation: \wss{The database should have the correct record according to the testing cases}
					
How test will be performed: Turn on the system. Start running the system with the different two objects showed up in the environment. Then check the record in database, it should have two different records of the objects.

				

\end{enumerate}


\paragraph{Automatic Testing}

\begin{enumerate}

\item{IPR6-1\\}

Control: Automatic
					
Initial State:The system is turned on, camera works properly. 
					
Input: The automatic testing tool will activate the photo storage function. 
					
Output: \wss{The photos taken by the camera with create date/time}

Test Case Derivation: \wss{The photo should have the time information}
					
How test will be performed:
Turn on the system, make sure the camera working as required. Then run the automatic testing tool, the tool will force the system to store the photos with create data/time information with it.


\item{IPR7-1\\}

Control: Automatic
					
Initial State:The system is turned on.
					
Input: The automatic testing tool will call the data storage module and send the object information to the module. 
					
Output: \wss{The different objects information will be stored in the database with unique ID}

Test Case Derivation: \wss{The object information in the database should have unique IDs}
					
How test will be performed:
Turn on the system. Then run the automatic testing tool, the tool will call the data storage module and pass the parameter into the module. Then the module will store the objects information with unique IDs.


\item{IPR8-1\\}

Control: Automatic
					
Initial State:The system is turned on.
					
Input: The automatic testing tool will call the photo storage function and send the photo to the module. 
					
Output: \wss{The photos in the data storage module should be in ascending or descending order of time}

Test Case Derivation: \wss{The photos in the database should be sorted}
					
How test will be performed:
Turn on the system. Then run the automatic testing tool, the tool will call the data storage module and pass the photos and parameter into the module. Then the module will sort the photos according to the create time.


\item{IPR9-1\\}

Control: Automatic
					
Initial State:The system is turned on.
					
Input: The automatic testing tool will call the photo storage function and send the photo to the module. 
					
Output: \wss{The photos in the data storage module should be in ascending or descending order of objects IDs}

Test Case Derivation: \wss{The photos in the database should be sorted}
					
How test will be performed:
Turn on the system. Then run the automatic testing tool, the tool will call the data storage module and pass the photos and parameter into the module. Then the module will sort the photos according to the objects IDs.


				

\end{enumerate}


\subsubsection{UI Interface Menu}


\paragraph{Manual Testing}

\begin{enumerate}

\item{UIR1-1\\}

Control: Manual
					
Initial State: Turn on the system and User interface show up in the laptop
					
Input: User's manipulation to the user interface
					
Output: \wss{The graphical display to the user}

Test Case Derivation: \wss{After user highlight a certain photo, the item on the display should change to another color.}
					
How test will be performed: 
	User turn on the system and user interface. Operate the system to highlight a object.		
	
	
\item{UIR2-1\\}

Control: Manual
					
Initial State: Turn on the system and User interface show up in the laptop
					
Input: User's manipulation to the user interface
					
Output: \wss{The graphical display to the user}

Test Case Derivation: \wss{After user change the sorting method, the system should give the certain response}
					
How test will be performed: 
	User turn on the system and user interface. Operate the system to change the sorting method.


\item{UIR3-1\\}

Control: Manual
					
Initial State: Turn on the system and User interface show up in the laptop
					
Input: User change the signal strength of WiFi
					
Output: \wss{The graphical display to the user}

Test Case Derivation: \wss{After user change the Wifi strength, the system should give the certain response}
					
How test will be performed: 
	User turn on the system and user interface. Switch the network connection to a weak strength Wifi, the system will give an alert.
	
\item{UIR4-1\\}

Control: Manual
					
Initial State: Turn on the system and User interface show up in the laptop
					
Input: User change the unplug the camera to insert a fault
					
Output: \wss{The graphical display to the user}

Test Case Derivation: \wss{After user unplug the camera, the system should give the response on status identifier}
					
How test will be performed: 
	User turn on the system and user interface. Unplug the camera, the system status identifier will change.
\end{enumerate}

\subsection{Tests for Nonfunctional Requirements}



\subsubsection{Usability}
		
\begin{enumerate}

\item{APR1-1\\}

Type: Structural, Manual
					
Initial State: \wss{The camera and its mount is connected to the laptop and stay stationary on a flat platform.}
					
Input/Condition: \wss{Launch the program normally and provide a physical impact during functioning}
					
Output/Result: \wss{No electronic components should be exposed and the shell should stay still}
					
How test will be performed: \wss{Test is proceeded during a normal working period where a physical impact will be applied to the hardware. The impact level is trying to simulate the damage that could possibily be done during normal functional period including dropping from the desk, over twisting the camera. After the impact, the appearance will be checked to see if the non-functional
requirements APR1 and APR2 are satisfied. There will be also be a survey question regarding if the product is physically safe in the sight}
					
\item{EUR1-1\\}

Type: Functional, Manual
					
Initial State: \wss{The program are stored properly in a USB and hardware parts are ready to be assembled}
					
Input: \wss{Users are asked to launch the program and connect the hardware }
					
Output: \wss{Users are able to successfully finish the set up and ready to use the system without any assistance}
					
How test will be performed: \wss{A group of people who do not have any electronics and coding background are asked to set up the system. Brief instructions regarding the functionality of the system will be given in advance and users will be asked to launch the program and connect the hardware. A survey question regarding the difficulty of setting up the system will be asked at the end of test. A score from 0 to 10 will be used to show user's experience. The majority of users should have no serious difficulty finishing the set up.}

\item{EUR2-1\\}

Type: Functional, Manual
					
Initial State: \wss{The program are set up properly and ready to use}
					
Input: \wss{Users are asked to use the program to find assigned object }
					
Output: \wss{Users are able to enter inputs into correct fields and proceed the function}
					
How test will be performed: \wss{A group of people who do not have any electronics and coding background are asked to use the system with brief instructions being told in advance. The number of people who successfully find the assigned object with the program will be noted. A survey question regarding the difficulty of using the program will be asked at the end of test.A score from 0 to 10 will be used to show user's experience. The majority of users should be able to clearly see input boxes and enter corresponding inputs.
}

\end{enumerate}

\subsubsection{Performance}
\begin{enumerate}


\item{SLR1-1\\}

Type: Functional, Manual
					
Initial State: \wss{The program are set up properly and ready to use, no further settings are required.}
					
Input: \wss{Information of the object is entered properly }
					
Output: \wss{The response time of the system to show the location of the object should be less than 5 second}
					
How test will be performed: \wss{Users will be asked to use the system to find different objects in the same assigned area and tester will record the time the system response. Then users will enter the same searching information while the objects are placed into another area (objects are all placed into the assigned area). Distance and number of items in the environment will be adjusted for each test condition. For objects with the largest distance and placed into the messiest environment , the response time must be below 5 seconds.
}


\item{SCR3-1\\}

Type: Functional, Manual
					
Initial State: \wss{The program are set up properly and ready to use, no further settings are required.}
					
Input: \wss{Information of two objects at the edge of the assigned area will be entered }
					
Output: \wss{The rotation speed where the camera rotate from one edge to the other is slow enough}
					
How test will be performed: \wss{Users will be asked to enter information of two objects at the edge of the assigned area one after another. As the area is assigned and the response time is restricted to less than 5 seconds, the speed that the camera rotate from edge to edge will be the maximum speed can be applied. The maximum rotation speed must be slow enough such that it will not 
cause any security concern.
}
\item{PAR1-1\\}

Type: Functional, Manual
					
Initial State: \wss{The program are set up properly and ready to use, no further settings are required.}
					
Input: \wss{The targer object will be moved one small step at a time }
					
Output: \wss{The location value displayed should always be whole number}
					
How test will be performed: \wss{Users will be asked to enter the information of one object and the tester will adjust the location of the object one small step at a time and observe the location value displayed in the program. Any value displayed in the program must be whole number and time value must be displayed with an accuracy of minute.
}

\item{RAR1-1\\}

Type: Functional, Manual
					
Initial State: \wss{The program are set up properly and ready to use, no further settings are required.}
					
Input: \wss{Part of the object to be searching for will be placed outside the assigned area }
					
Output: \wss{The device should return NOT FOUND if the portion of the object showing in the camera is not representative enough}
					
How test will be performed: \wss{User will be asked to enter the information of one object where part of it is not in the camera. As the rotation angle is restricted, the camera must not over-rotate to an unexpected angle. If the partial information is not representative, the program should return NOT FOUND.
}

\item{RFR2-1\\}

Type: Functional, Manual
					
Initial State: \wss{The program are set up properly and ready to use, no further settings are required.}
					
Input: \wss{wrong parameters will be entered into input boxes }
					
Output: \wss{The program will return error messages }
					
How test will be performed: \wss{User will be asked to enter wrong parameters in corresponding input boxes. For example, input number in color box. The program must present a error message notifying the user that wrong parameters are entered and clear all the input boxes.
}
\end{enumerate}


\subsection{Traceability Between Test Cases and Requirements}
\begin{center}
    \begin{tabular}{||c | c ||}
    \hline
    Requirements & Tests\\
    \hline
    IPR1&IPR1-1\\
    \hline
    IPR2&IPR2-1\\
    \hline
    IPR3&IPR3-1\\
    \hline
    IPR4&IPR4-1\\
    \hline
    IPR5&IPR5-1\\
    \hline
    IPR6&IPR6-1\\
    \hline
    IPR7&IPR7-1\\
    \hline
    IPR8&IPR8-1\\
    \hline
    IPR9&IPR9-1\\
    \hline
    UIR1&UIR1-1\\
    \hline
    UIR2&UIR2-1\\
    \hline
    UIR3&UIR3-1\\
    \hline
    UIR4&UIR4-1\\
    \hline
    APR1,APR2,SCR1 & APR1-1\\
    \hline
    EUR1,LER1,LER2 & EUR1-1\\
    \hline
    EUR2,UPR1,ACR1 & EUR2-1\\
    \hline
    SLR1 & SLR1-1\\
    \hline
    SCR3 & SCR3-1\\
    \hline
    PAR1,PAR2,PAR3 & PAR1-1\\
    \hline
    RAR1 & RAR1-1\\
    \hline
    RFR2 & RFR2-1\\
    \hline
    \end{tabular}
    
\end{center}

\section{Unit Test Description}

\wss{Reference your MIS and explain your overall philosophy for test case
  selection.}  
\wss{This section should not be filled in until after the MIS has
  been completed.}

\subsection{Unit Testing Scope}

\wss{What modules are outside of the scope.  If there are modules that are
  developed by someone else, then you would say here if you aren't planning on
  verifying them.  There may also be modules that are part of your software, but
  have a lower priority for verification than others.  If this is the case,
  explain your rationale for the ranking of module importance.}

\subsection{Tests for Functional Requirements}

\wss{Most of the verification will be through automated unit testing.  If
  appropriate specific modules can be verified by a non-testing based
  technique.  That can also be documented in this section.}

\subsubsection{Module 1}

\wss{Include a blurb here to explain why the subsections below cover the module.
  References to the MIS would be good.  You will want tests from a black box
  perspective and from a white box perspective.  Explain to the reader how the
  tests were selected.}

\begin{enumerate}

\item{test-id1\\}

Type: \wss{Functional, Dynamic, Manual, Automatic, Static etc. Most will
  be automatic}
					
Initial State: 
					
Input: 
					
Output: \wss{The expected result for the given inputs}

Test Case Derivation: \wss{Justify the expected value given in the Output field}

How test will be performed: 
					
\item{test-id2\\}

Type: \wss{Functional, Dynamic, Manual, Automatic, Static etc. Most will
  be automatic}
					
Initial State: 
					
Input: 
					
Output: \wss{The expected result for the given inputs}

Test Case Derivation: \wss{Justify the expected value given in the Output field}

How test will be performed: 

\item{...\\}
    
\end{enumerate}

\subsubsection{Module 2}

...

\subsection{Tests for Nonfunctional Requirements}

\wss{If there is a module that needs to be independently assessed for
  performance, those test cases can go here.  In some projects, planning for
  nonfunctional tests of units will not be that relevant.}

\wss{These tests may involve collecting performance data from previously
  mentioned functional tests.}

\subsubsection{Module ?}
		
\begin{enumerate}

\item{test-id1\\}

Type: \wss{Functional, Dynamic, Manual, Automatic, Static etc. Most will
  be automatic}
					
Initial State: 
					
Input/Condition: 
					
Output/Result: 
					
How test will be performed: 
					
\item{test-id2\\}

Type: Functional, Dynamic, Manual, Static etc.
					
Initial State: 
					
Input: 
					
Output: 
					
How test will be performed: 

\end{enumerate}

\subsubsection{Module ?}

...

\subsection{Traceability Between Test Cases and Modules}

\wss{Provide evidence that all of the modules have been considered.}
				
\bibliographystyle{plainnat}

\bibliography{../../refs/References}

\newpage

\section{Appendix}

This is where you can place additional information.

\subsection{Symbolic Parameters}

The definition of the test cases will call for SYMBOLIC\_CONSTANTS.
Their values are defined in this section for easy maintenance.

\subsection{Usability Survey Questions?}

\wss{1.Do you feel unsafe when you see the product in the first sight?(for example, any exposed electronics and sharp corner that makes you feel uncomfortable)\\
\bigskip\\
0 || 1 || 2 || 3 || 4 || 5 || 6 || 7 || 8 || 9 || 10\\
0 represents very unsafe \space 10 represents very safe\\
\bigskip\\
2.Do you have difficulty when setting up the system (launch the program and connect the hardware)?\\
\bigskip\\
0 || 1 || 2 || 3 || 4 || 5 || 6 || 7 || 8 || 9 || 10\\
0 represents very difficult \space 10 represents very easy\\
\bigskip\\
3.Do you have difficulty when using the program? (knowing where to put corresponding inputs and found interface is easy to understand)\\
\bigskip\\
0 || 1 || 2 || 3 || 4 || 5 || 6 || 7 || 8 || 9 || 10\\
0 represents very difficult \space 10 represents very easy\\


}

\newpage{}
\section*{Appendix --- Reflection}

The information in this section will be used to evaluate the team members on the
graduate attribute of Lifelong Learning.  Please answer the following questions:

\begin{enumerate}
  \item 
  \item 
\end{enumerate}

\end{document}
