\documentclass[12pt, titlepage]{article}

\usepackage{booktabs}
\usepackage{tabularx}
\usepackage{hyperref}
\hypersetup{
    colorlinks,
    citecolor=blue,
    filecolor=black,
    linkcolor=red,
    urlcolor=blue
}
\usepackage[round]{natbib}

%% Comments

\usepackage{color}

\newif\ifcomments\commentstrue %displays comments
%\newif\ifcomments\commentsfalse %so that comments do not display

\ifcomments
\newcommand{\authornote}[3]{\textcolor{#1}{[#3 ---#2]}}
\newcommand{\todo}[1]{\textcolor{red}{[TODO: #1]}}
\else
\newcommand{\authornote}[3]{}
\newcommand{\todo}[1]{}
\fi

\newcommand{\wss}[1]{\authornote{blue}{SS}{#1}} 
\newcommand{\plt}[1]{\authornote{magenta}{TPLT}{#1}} %For explanation of the template
\newcommand{\an}[1]{\authornote{cyan}{Author}{#1}}

%% Common Parts

\newcommand{\progname}{Mechtronics Enigeering} % PUT YOUR PROGRAM NAME HERE
\newcommand{\authname}{Team 32, Wingman
\\ Edward He
\\ Erping Zhang
\\ Guangwei Tang
\\ Peng Cui
\\ Peihua Jin } % AUTHOR NAMES                  

\usepackage{hyperref}
    \hypersetup{colorlinks=true, linkcolor=blue, citecolor=blue, filecolor=blue,
                urlcolor=blue, unicode=false}
    \urlstyle{same}
                                


\begin{document}

\title{Project Title: System Verification and Validation Plan for \progname{}} 
\author{\authname}
\date{\today}
	
\maketitle

\pagenumbering{roman}

\section{Revision History}

\begin{tabularx}{\textwidth}{p{3cm}p{3cm}X}
\toprule {\bf Date} & {\bf Version} & {\bf Notes}\\
\midrule
2022-11-2 & Edward He, Erping Zhang & Revision 0\\
& Guangwei Tang, Peng Cui & \\
& Peihua Jin & \\
\\
2022-3-15 & Peihua Jin & Revision 1.0, re-edit the tests requirements according to the Rev0 demo\\
\\
2022-3-18 & Peihua Jin & Revision 1.1, add new tests requirement for Functional Requirements\\
\\
2022-3-22 & Peng Cui & Revision 1.2, describe certain tests in a more specific way according to the feedback, such as adding units and ranges\\
\\
2022-4-2 & Peihua Jin & Revision 1.3, revise RAR1-1 to match with SRS doc, revise the table for traceability\\



\bottomrule
\end{tabularx}

\newpage

\tableofcontents

\listoftables


\newpage

\section{Symbols, Abbreviations and Acronyms}

\renewcommand{\arraystretch}{1.2}
\begin{tabular}{l l} 
  \toprule		
  \textbf{symbol} & \textbf{description}\\
  \midrule 
  SRS & Software Requirements Specifications\\
  MG & Module Guide\\
  MIS & Module Interface Specification\\
  \bottomrule
\end{tabular}\\



\newpage

\pagenumbering{arabic}


\section{General Information}

\subsection{Summary}

The SmartVault project is a Mechatronics system that is able to assist user identify the location of assigned objects in a given area.

\subsection{Objectives}

This document is intended to develop a systematic plan for testing the functionality of the system.It meant to show the system has met the requirements in both software and hardware aspects mentioned in requirements document.In this document, system test and unit test will be conducted to check the functionality of the system. Functional and non-functional requirements will be separated for testing for both section. By the end of testing process, it can be shown that the system is working properly and available for usage.

\subsection{Relevant Documentation}

\begin{enumerate}
    \item \href{https://github.com/Edwardhyw/smartVault/tree/main/docs/DevelopmentPlan}{Development Plan}
    \item \href{https://github.com/Edwardhyw/smartVault/tree/main/docs/SRS}{System Requirements Specification} 
    \item \href{https://github.com/Edwardhyw/smartVault/tree/main/docs/HazardAnalysis}{Hazard Analysis} 
    \item \href{https://github.com/Edwardhyw/smartVault/tree/main/docs/Design/MIS}{Module Interface Specification} 
    \item \href{https://github.com/Edwardhyw/smartVault/tree/main/docs/Design/MG}{Module Guide} 
\end{enumerate}

\section{Plan}
In this section, it provides an overview and roadmap for the document. This section outlines the members of the verification and validation team for the project SmartVault and their assigned roles. This section describes the methods to verify and validate each component of the project, which includes SRS, Design, implementation, testing tools and the software for SmartVault.

\subsection{Verification and Validation Team}

\begin{table}[h]
\begin{center}
\begin{tabular}{|l | l|}
\hline
  \textbf{Member} & \textbf{Role}\\
  \hline
  Edward He & Static Verification Lead; Back-end Testing\\
  \hline
  Erping Zhang & Design Verification Lead; Front-end Testing \\
  \hline
  Guangwei Tang & SRS Verification Lead; Hardware Testing\\
  \hline
  Peng Cui & Dynamic Verification Lead; Code Inspection\\
  \hline
  Peihua Jin & Automated Testing Lead; Code Inspection\\
  \hline
  Spencer Smith & Review of documents and system\\
  \hline
  Peer Groups & Review of documents and system\\
  \hline
\end{tabular}
\end{center}
\caption{Verification and Validation Team Members and Roles}            

\end{table}
\break
\subsection{SRS Verification Plan}

In order to verify the SRS document, the SRS checklist will be used to make sure that the document included all components on the checklist and are complete. The SRS document will also be reviewed by other groups and based on their feedbacks, changes will be made accordingly. Professor Spencer Smith and the teaching team will also review the document and provide feedbakcs. Requirements Listed in the SRS document will be verified and validated by the methods listed in the following sections.



\subsection{Design Verification Plan}

To verify the design of the system, MG and MIS checklist will be utilized. The team has decided and implemented a procedure for each individual design made by different team developer. Any design decision is required to be reviewed and verified by at least 3 other members. The design of SmartVault will also be reviewed by peers and instructor. 

\subsection{Implementation Verification Plan}

The implementation of the SmartVault will be done with both dynamic and static verification. For dynamic verification, both non-functional and functional requirements implementation will be tested. The test will consist both system test and unit test to ensure the written code comply with the requirements and functionalities. More details for the dynamic verification refer to section 5 and 6 of this document. For the static verification, it will be done with code inspection and coding standards. The team will utilize  Git and Github to inspect and verify coding. The merge of a new functionality or modification of current implementation has to be reviewed by 2 others member prior to merging to the main branch. Code inspections/walkthrough will be done weekly to ensure it follows the coding standards. SmartVault will be primarily written in Python and team members are required to follow PEP8 standard by using linter Pylint.
\subsection{Automated Testing and Verification Tools}

\begin{itemize}
    \item Coding Standard(PEP 8) : pylint
    \item Unit Testing Frameworks: PyUnit
   
   
\end{itemize}

\subsection{Software Validation Plan}

Software validation will be mostly conducted by team members. External validation is set to be conducted with consumers(stakeholders) of the product where they can try out the system and report if there is anything they would like to change.

\section{System Test Description}
	
\subsection{Tests for Functional Requirements}

\subsubsection{Image Processing and Storage Functional Requirements}
		
\paragraph{Manual Testing}

\begin{enumerate}

\item{IPR1-1\\}

Control: Manual
					
Initial State: The system is turned on, the working environment should be empty.
					
Input: Images of the working environment and a human shows up in the environment.
					
Output: Coordinate of the detected human body and an output information as "human coming".

Test Case Derivation: The output coordinate of the detected human body should located on the human body in the image.
					
How test will be performed: Turn on the system. Start running the system without the human showed up in the environment. Then the human should move into the environment and the system will put a square shape on the human body in the image.

\item{IPR1-2\\}

Control: Manual
					
Initial State: The system is turned on, the working environment should be empty.
					
Input: Images of the working environment and a human walks around the frame.
					
Output: Coordinate of the detected human body.

Test Case Derivation: The output coordinate of the detected human body should keep updating the location on the human body in the image. And the camera should keep following the human movement.
					
How test will be performed: Turn on the system. Start running the system with the human showed up in the environment. Then the human should walk around the environment and the system will put a square shape on the human body in the image.

\item{IPR1-3\\}

Control: Manual
					
Initial State: The system is turned on, the working environment should be empty
					
Input: Images of the working environment and a human leaves the frame
					
Output: An output information as "human leaving".

Test Case Derivation: The output coordinate of the detected human body should stop updating once human leaves the frame.
					
How test will be performed: Turn on the system. Start running the system with the human showed up in the environment. Then the human should leave the environment and the system will wait for a certain period.


\item{IPR2-1\\}

Control: Manual
					
Initial State: The system is turned on, the working environment includes at least one item.
					
Input: Images of the working environment and a new object was placed in the environment.
					
Output: A image highlighted with newly-added objects after the process of image subtraction being applied .

Test Case Derivation: The movement shall be captured and the differences between the two frames will be illustrated.
					
How test will be performed: Turn on the system. The tool will call the image subtraction algorithms and pass the photos and parameter into the module. Then the module will generate the image highlighted with desired differences

\item{IPR2-2\\}

Control: Manual
					
Initial State: The system is turned on, the working environment includes at least one item.
					
Input: Images of the working environment and object being removed out of the environment.
					
Output: A image with nothing after the process of image subtraction being applied .

Test Case Derivation: The movement shall be captured and the differences between the two frames will be illustrated.
					
How test will be performed: Turn on the system. The tool will call the image subtraction algorithms and pass the photos and parameter into the module. Then the module will generate the image highlighted with desired differences

\item{IPR2-3\\}

Control: Manual
					
Initial State: The system is turned on, the working environment includes at least one item.
					
Input: Images of the working environment and object being moved within the environment.
					
Output: A image updated with moved objects after the process of image subtraction being applied .

Test Case Derivation: The movement shall be captured and the differences between the two frames will be illustrated.
					
How test will be performed: Turn on the system. The tool will call the image subtraction algorithms and pass the photos and parameter into the module. Then the module will generate the image highlighted with desired differences

\item{IPR3-1\\}

Control: Manual
					
Initial State: The system is turned on, and human starts to leave the working environment.
					
Input: Images of the working environment and the system detected human's leaving.
					
Output: The mounted camera shall be back to the default position, and takes a photo after 5 seconds.

Test Case Derivation: The camera will smoothly rotate back to the default position and does not affect the image subtraction.
					
How test will be performed: Turn on the system. The tool will call the picturing function and pass the photos after human's leaving. Then the module will generate a photo and temporarily save it for further comparisons during the test.

\item{IPR4-1\\}

Control: Manual
					
Initial State: The system is turned on, the working environment should include at least one item.
					
Input: Images of the working environment and object in the environment. Different Images should have different location of the object in the environment.
					
Output: The re-location mode should be activated.

Test Case Derivation: The re-location mode should be activated when the object location change.
					
How test will be performed: Turn on the system. Start running the system with the objects showed up in the environment. Then change the location of object in the environment, the system should activated the re-location mode.


\item{IPR5-1\\}

Control: Manual
					
Initial State:The system is turned on, camera works properly. 
					
Input: The automatic testing tool will activate the photo storage function. 
					
Output: The photos taken by the camera with create date/time.

Test Case Derivation: The photo should have the detailed time information.
					
How test will be performed: Turn on the system. Start running the fileStorage function with the parameter, 'i', as insertion. Then check the record in the local folder, it should store two photos in total, one with create data/time information shall be inside the 'item' folder, and the other will be inside the 'location' folder.



\end{enumerate}


\paragraph{Automatic Testing}

\begin{enumerate}

\item{IPR6-1\\}

Control: Automatic
					
Initial State: The system is turned on, the working environment should include at least two items.
					
Input: Images of the working environment and at least two different objects in the environment. 
					
Output: The system should only make any update inside the location folder.

Test Case Derivation: The folder should have the correct updating according to the testing cases.
					
How test will be performed: Turn on the system. Start running the fileStorage function with the parameter, 'u', as updating. Then check the record in the local folder, it should only update the 'location' folder for the desired item.

				


\item{IPR7-1\\}

Control: Automatic
					
Initial State: The system is turned on.
					
Input: The automatic testing tool will call the file storage module and send the object information to the module. 
					
Output: The different objects information will be stored in the folder with unique ID.

Test Case Derivation: The object information in the file folder should have unique IDs.
					
How test will be performed:
Turn on the system. Then run the automatic testing tool, the tool will call the data storage module and pass the parameter into the module. Then the module will store the objects information with unique IDs.


\item{IPR8-1\\}

Control: Automatic
					
Initial State:The system is turned on.
					
Input: The automatic testing tool will call the photo storage function and send the photo to the module. 
					
Output: The photos in the data storage module should be in ascending or descending order of time.

Test Case Derivation: The photos in the file folder should be sorted.
					
How test will be performed:
Turn on the system. Then run the automatic testing tool, the tool will call the data storage module and pass the photos and parameter into the module. Then the module will sort the photos according to the create time.


\item{IPR9-1\\}

Control: Automatic
					
Initial State: The system is turned on.
					
Input: The automatic testing tool will call the photo storage function and send the photo to the module. 
					
Output: The photos in the data storage module should be in ascending or descending order of objects IDs.

Test Case Derivation: The photos in the file folder should be sorted.
					
How test will be performed:
Turn on the system. Then run the automatic testing tool, the tool will call the data storage module and pass the photos and parameter into the module. Then the module will sort the photos according to the objects IDs.


				

\end{enumerate}


\subsubsection{UI Menu Functional Requirements}


\paragraph{Manual Testing}

\begin{enumerate}

\item{UIR1-1\\}

Control: Manual
					
Initial State: Turn on the system and User interface show up in the device.
					
Input: User's manipulation to the user interface.
					
Output: The graphical displays to the user.

Test Case Derivation: After user highlight a certain photo, the item on the display should change to another color.
					
How test will be performed: User turns on the system and user interface. Operate the system to highlight a object.		
	
	
\item{UIR2-1\\}

Control: Manual
					
Initial State: Turn on the system and User interface show up in the device.
					
Input: User's manipulation to the user interface.
					
Output: The graphical displays to the user.

Test Case Derivation: After user change the sorting method, the system should give the certain response.
					
How test will be performed: User turns on the system and user interface. Operate the system to change the sorting method.

\item{UIR3-1\\}

Control: Manual
					
Initial State: Turn on the system and User interface show up in the device.
					
Input: User types in his/her username and password correctly.
					
Output: The graphical displays to the user.

Test Case Derivation: After user clicks the 'login in' button, the system should give the certain response. He/She will not be allowed to access the application unless the username and password are correct.
					
How test will be performed: User turns on the system and user interface. Try to type out the personal username and password. Once both of them are correct, he/she will successfully get into the main menu.

\item{UIR3-2\\}

Control: Manual
					
Initial State: Turn on the system and User interface show up in the device.
					
Input: User types in his/her username and password incorrectly.
					
Output: The graphical displays to the user.

Test Case Derivation: After user clicks the 'login in' button, he/she will be notified that the username or the password is wrong. User shall then retry the login process until the username and password are correct.
					
How test will be performed: User turns on the system and user interface. Try to type out the wrong username and password. He/She will have no access into the application.
	
\item{UIR4-1\\}

Control: Manual
					
Initial State: Turn on the system and User interface show up in the device.
					
Input: User changes the unplugs the camera to insert a fault.
					
Output: The graphical displays to the user.

Test Case Derivation: After user unplug the camera, the system should give the response on status identifier.
					
How test will be performed: User turns on the system and user interface. Unplug the camera, the system status identifier will change.

\item{UIR5-1\\}

Control: Manual
					
Initial State: Turn on the system and User interface show up in the device.
					
Input: User presses the Technical Support button.
					
Output: The Technical Support Window is shown.

Test Case Derivation: After pressing the button, the user can see the technical support window.
					
How test will be performed: 
User turns on the system and user interface. After pressing the technical support window, the UI shows a window  with contact information of the program developers. 

\end{enumerate}

\subsection{Tests for Nonfunctional Requirements}




\subsubsection{Usability}
		
\begin{enumerate}

\item{APR1-1\\}

Type: Structural, Manual
					
Initial State: The camera and its mount is connected to the device and stay stationary on a flat platform.
					
Input/Condition: Launch the program normally and provide a physical impact during functioning.
					
Output/Result: No electronic components should be exposed and the shell should stay still.
					
How test will be performed: Test is proceeded during a normal working period where a physical impact will be applied to the hardware. The impact level is trying to simulate the damage that could possibly be done during normal functional period including dropping from the desk, over twisting the camera. After the impact, the appearance will be checked to see if the non-functional
requirements APR1 and APR2 are satisfied. There will be also be a survey question regarding if the product is physically safe in the sight
					
\item{EUR1-1\\}

Type: Functional, Manual
					
Initial State: The program is packed properly as an executable file and hardware parts are ready to be assembled.
					
Input: Users are asked to launch the program and connect the hardware. 
					
Output: Users are able to successfully finish the set up and ready to use the system without any assistance.
					
How test will be performed: A group of people who do not have any electronics and coding background are asked to set up the system. Brief instructions regarding the functionality of the system will be given in advance and users will be asked to launch the program and connect the hardware. A survey question regarding the difficulty of setting up the system will be asked at the end of test. A score from 0 to 10 will be used to show user's experience. The majority of users should have no serious difficulty finishing the set up.

\item{EUR2-1\\}

Type: Functional, Manual
					
Initial State: The program is set up properly and ready to use.
					
Input: Users are asked to use the program to find assigned object.
					
Output: Users are able to enter inputs into correct fields and proceed the function.
					
How test will be performed: A group of people who do not have any electronics and coding background are asked to use the system with brief instructions being told in advance. The number of people who successfully find the assigned object with the program will be noted. A survey question regarding the difficulty of using the program will be asked at the end of test.A score from 0 to 10 will be used to show user's experience. The majority of users should be able to clearly see input boxes and enter corresponding inputs.


\end{enumerate}

\subsubsection{Performance}
\begin{enumerate}


\item{SLR1-1\\}

Type: Functional, Manual

Unit: Seconds
					
Initial State: The program is set up properly and ready to use, no further settings are required.
					
Input: Information of the object is entered properly.
					
Output: The response time of the system to show the location of the object should be less than 5 second.
					
How test will be performed: Users will be asked to use the system to find different objects in the same assigned area and tester will record the time the system response. Then users will enter the same searching information while the objects are placed into another area (objects are all placed into the assigned area). Distance and number of items in the environment will be adjusted for each test condition. For objects with the largest distance and placed into the messiest environment , the response time must be below 5 seconds.



\item{SCR3-1\\}

Type: Functional, Manual

Units: Degree/seconds

Ranges: -30 to 30 degree/seconds
					
Initial State: The program is set up properly and ready to use, no further settings are required.
					
Input: Information of two objects at the edge of the assigned area will be entered.
					
Output: The rotation speed where the camera rotates from one edge to the other is slow and smooth.
					
How test will be performed: Users will be asked to enter information of two objects at the edge of the assigned area one after another. As the area is assigned and the response time is restricted to less than 5 seconds, the speed that the camera rotate from edge to edge will be the maximum speed can be applied. The maximum rotation speed must be slow enough such that it will not 
cause any security concern.

\item{PAR2-1\\}

Type: Functional, Manual
					
Initial State: The program are set up properly and ready to use, no further settings are required.
					
Input: The target object will be moved one small step at a time.
					
Output: The location value displayed should always be whole number.
					
How test will be performed: Users will be asked to enter the information of one object and the tester will adjust the location of the object one small step at a time and observe the location value displayed in the program. Any value displayed in the program must be whole number and time value must be displayed with an accuracy of minute.


\item{RAR1-1\\}

Type: Functional, Manual

Units: Degree

Ranges: -180 to 180 degree
					
Initial State: The program is set up properly and ready to use, no further settings are required.
					
Input: The camera will keep rotating. 
					
Output: The camera would never exceed the range of -180 degrees and +180 degrees.
					
How test will be performed: Human will be asked to walk around the testing environment. Once the angle reaches -180 or 180 degrees, the camera will stop rotating.


\item{RFR2-1\\}

Type: Functional, Manual
					
Initial State: The program is set up properly and ready to use, no further settings are required.
					
Input: wrong parameters will be entered into input boxes 
					
Output: The program will return error messages 
					
How test will be performed: User will be asked to enter wrong parameters in corresponding input boxes. For example, input number in color box. The program must present a error message notifying the user that wrong parameters are entered and clear all the input boxes.

\end{enumerate}


\subsection{Traceability Between Test Cases and Requirements}
\begin{center}
    \begin{tabular}{||c | c ||}
    \hline
    Requirements & Tests\\
    \hline
    IPR1&IPR1-1\\
    \hline
    IPR1&IPR1-2\\
    \hline
    IPR1&IPR1-3\\
    \hline
    IPR2&IPR2-1\\
    \hline
    IPR2&IPR2-2\\
    \hline
    IPR2&IPR2-3\\
    \hline
    IPR3&IPR3-1\\
    \hline
    IPR4&IPR4-1\\
    \hline
    IPR5&IPR5-1\\
    \hline
    IPR6&IPR6-1\\
    \hline
    IPR7&IPR7-1\\
    \hline
    IPR8&IPR8-1\\
    \hline
    IPR9&IPR9-1\\
    \hline
    UIR1&UIR1-1\\
    \hline
    UIR2&UIR2-1\\
    \hline
    UIR3&UIR3-1\\
    \hline
    UIR3&UIR3-2\\
    \hline
    UIR4&UIR4-1\\
    \hline
    UIR5&UIR5-1\\
    \hline
    APR1,APR2,SCR1 & APR1-1\\
    \hline
    EUR1,LER1,LER2 & EUR1-1\\
    \hline
    EUR2,UPR1,ACR1 & EUR2-1\\
    \hline
    SLR1 & SLR1-1\\
    \hline
    SCR3 & SCR3-1\\
    \hline
    PAR1,PAR2,PAR3 & PAR2-1\\
    \hline
    RAR1 & RAR1-1\\
    \hline
    RFR2 & RFR2-1\\
    \hline
    \end{tabular}
    
\end{center}

\section{Proof of Concept Testing}
Before any serious development of the game starts, a proof-of-concept test would be carried out to show the undertaking is feasible. The remaining of this section describes the proof-of-concept test in detail.

\subsection{Significant Risks}
The successful completion of the project depends on overcoming the following significant risks:
\begin{enumerate}
\item{}
In order to satisfy the functionality of human detection, the camera must be able to rotate and follow the movement of the user. The risk is to make the user located in the center of the screen all the time. Otherwise, the observation may behave poorly and unclearly if the user is out of the screen or located at the edge of the screen.			
\item{}
SmartVault is used to keep the items of users or to find out when and what gets lost or missed. The most important part of the project is to identify the differences through the compare between two images.
\end{enumerate}
\subsection{Demonstration Plan}
The prototype will try to test human detection firstly. Three cases will be applied for the proof-of-concept testing.
\begin{itemize}
\item{test \#1\\}
Type: Functional, automatic\\
Initial State: User stays in the center of the screen\\
Input: Null\\
Output: The camera does not rotate, and the user is identified with a yellow rectangle shown in the screen.\\			
\item{test \#2\\}
Type: Functional, dynamic, automatic\\
Initial State: User stays at the edge of the screen\\
Input: Null\\
Output: The user is identified with a yellow rectangle shown in the screen, then the camera begins to rotate until the user is in the center of the screen.\\
\item{test \#3\\}
Type: Functional, automatic\\
Initial State: User is out of the screen\\
Input: Null\\
Output: The camera does not rotate, and the user cannot be identified on the screen.\\
\item{test \#4\\}
Type: Functional, dynamic, automatic\\
Initial State: User is in the screen\\
Input: User randomly walk around the room\\
Output: The camera rotates and follows the user, and the user is identified with a yellow rectangle shown in the screen. In addition, the location of the rectangle also moves depending on the movement of the user.\\
\end{itemize}
The prototype will try to test item identification secondly. Four cases will be applied for the proof-of-concept testing.
\begin{itemize}
\item{test \#5\\}
Type: Functional, dynamic, automatic\\
Initial State: Null\\
Input: Item A being newly added into the working environment.\\
Output: SmartVault will first compare two frames and record the changes. Item A will be recorded in 'item' folder with detailed date and time, its ID, item\_0 in this case, will be stored in 'location' folder.\\ 
Goal: To demonstrate the function for the insertion of new item.\\
\item{test \#6\\}
Type: Functional, dynamic, automatic\\
Initial State: Null\\
Input: Two frames with the same item, but the location of item A is changed.\\
Output: The difference of the location is identified. SmartVault will record these change and upload to the local file. The only change will be updated in 'location' folder.\\ 
Goal: To demonstrate the function for updating the recorded items.\\
\item{test \#7\\}
Type: Functional, dynamic, automatic\\
Initial State: Null\\
Input: Item B being newly added into the working environment.\\
Output: SmartVault will first compare two frames and record the changes. In addition, SmartVault would compare item B with all recorded items inside the list, item A in this case, and try to find as many match points as possible. Again, item B will be recorded in 'item' folder with detailed date and time, its ID, item\_1 in this case, will be stored in 'location' folder.\\ 
Goal: To demonstrate the function for differentiate different items.\\
\item{test \#8\\}
Type: Functional, dynamic, automatic\\
Initial State: Null\\
Input: Both item A and B will be taken away from the testing environment.\\
Output: SmartVault will identify the changes and update new frames respectively into 'location' folder. In other words, both of item\_0 and item\_1 will be shown as being already taken away.\\
Goal: To demonstrate the function for capturing the movement of two items concurrently.\\
\end{itemize}

\section{Unit Test Description}

\subsection{Unit Testing of Internal Functions}
In order to build unit tests for the internal functions of the program, certain modules which can return values could be tested. This will involve choosing proper method and giving them input values. Through comparing what they are expected to output and the actual outputs, a series of unit tests can be created. These unit tests will choose both proper inputs and those could generate exceptions. The acceptable range for the unit test could be determined through testing the output with exceptions. There is no need for stubs or drivers to be applied for the testing. Since the tests have been already done by the individual classes. Some coverage metrics will be shown to measure how much of the code had been covered for testing purposes. The goal is to reach as much as possible to ensure that all functions related to various modules get tested adequately. The expectation of the coverage percentage of unit tests is set to 80 percent.
\subsection{Unit Testing of Output Files}
In order to create unit tests for the output files, the quality and accuracy of the output images will be considered for testing. This will involve creating certain situation and compare the actual results with the expected ones. For the saved images in the output files, they will be rated depending on how similar they are with the expected images. Only those output images being rated with a score larger than 95 percent will be considered as qualified. On the other hand, if the score does not meet 95 percent, the test case fails and the developer team should analysis the reason behind it and redo the test case. In addition, some extreme cases will be applied to the unit testing process, to measure the maximum and minimum acceptable range for the unit tests.

				
\bibliographystyle{plainnat}

\bibliography{../../refs/References}

\newpage

\section{Appendix}

\subsection{Usability Survey Questions}

1.Do you feel unsafe when you see the product in the first sight?(for example, any exposed electronics and sharp corner that makes you feel uncomfortable\\
\bigskip\\
0 || 1 || 2 || 3 || 4 || 5 || 6 || 7 || 8 || 9 || 10\\
0 represents very unsafe \space 10 represents very safe\\
\bigskip\\
2.Do you have difficulty when setting up the system (launch the program and connect the hardware)?\\
\bigskip\\
0 || 1 || 2 || 3 || 4 || 5 || 6 || 7 || 8 || 9 || 10\\
0 represents very difficult \space 10 represents very easy\\
\bigskip\\
3.Do you have difficulty when using the program? (knowing where to put corresponding inputs and found interface is easy to understand)\\
\bigskip\\
0 || 1 || 2 || 3 || 4 || 5 || 6 || 7 || 8 || 9 || 10\\
0 represents very difficult \space 10 represents very easy\\






\end{document}
