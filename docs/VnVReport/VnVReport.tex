\documentclass[12pt, titlepage]{article}

\usepackage{booktabs}
\usepackage{tabularx}
\usepackage{hyperref}
\hypersetup{
    colorlinks,
    citecolor=black,
    filecolor=black,
    linkcolor=red,
    urlcolor=blue
}
\usepackage[round]{natbib}

%% Comments

\usepackage{color}

\newif\ifcomments\commentstrue %displays comments
%\newif\ifcomments\commentsfalse %so that comments do not display

\ifcomments
\newcommand{\authornote}[3]{\textcolor{#1}{[#3 ---#2]}}
\newcommand{\todo}[1]{\textcolor{red}{[TODO: #1]}}
\else
\newcommand{\authornote}[3]{}
\newcommand{\todo}[1]{}
\fi

\newcommand{\wss}[1]{\authornote{blue}{SS}{#1}} 
\newcommand{\plt}[1]{\authornote{magenta}{TPLT}{#1}} %For explanation of the template
\newcommand{\an}[1]{\authornote{cyan}{Author}{#1}}

%% Common Parts

\newcommand{\progname}{Mechtronics Enigeering} % PUT YOUR PROGRAM NAME HERE
\newcommand{\authname}{Team 32, Wingman
\\ Edward He
\\ Erping Zhang
\\ Guangwei Tang
\\ Peng Cui
\\ Peihua Jin } % AUTHOR NAMES                  

\usepackage{hyperref}
    \hypersetup{colorlinks=true, linkcolor=blue, citecolor=blue, filecolor=blue,
                urlcolor=blue, unicode=false}
    \urlstyle{same}
                                


\begin{document}

\title{Verification and Validation Report: \progname} 
\author{\authname}
\date{\today}
	
\maketitle

\pagenumbering{roman}

\section{Revision History}

\begin{tabularx}{\textwidth}{p{3cm}p{2cm}X}
\toprule {\bf Date} & {\bf Version} & {\bf Notes}\\
\midrule
Date 1 & 1.0 & Notes\\
Date 2 & 1.1 & Notes\\
\bottomrule
\end{tabularx}

~\newpage

\section{Symbols, Abbreviations and Acronyms}

\renewcommand{\arraystretch}{1.2}
\begin{tabular}{l l} 
  \toprule		
  \textbf{symbol} & \textbf{description}\\
  \midrule 
  T & Test\\
  \bottomrule
\end{tabular}\\

\wss{symbols, abbreviations or acronyms -- you can reference the SRS tables if needed}

\newpage

\tableofcontents

\listoftables %if appropriate

\listoffigures %if appropriate

\newpage

\pagenumbering{arabic}

This document ...

\section{Functional Requirements Evaluation}
\begin{table}[h]
\begin{center}
\begin{tabular}{|p{0.25\textwidth} | p{0.75\textwidth}|}
\hline
  Test Number & IPR1-1\\
  \hline
  Requirement Reference & IPR1\\
  \hline
  Requirement &  The system should be able to identify human’s body\\
  \hline
  Input & Images of the working environment and a human show up in
the environment\\
  \hline
  Desired Output & Coordinate of the detected human body\\
  \hline
  Actual Output & Correct coordinate of the detected human body\\
  \hline
  Conclusion & The test pass as expected\\
  \hline
\end{tabular}
\end{center}           
\end{table}

\begin{table}[h]
\begin{center}
\begin{tabular}{|p{0.25\textwidth} | p{0.75\textwidth}|}
\hline
  Test Number & UIR4-1\\
  \hline
  Requirement Reference & UIR4\\
  \hline
  Requirement &  The UI must be able to allow the user to view the system’s status at any given point in time.\\
  \hline
  Input & User change the unplug the camera to insert a fault\\
  \hline
  Desired Output & The graphical display to the user\\
  \hline
  Actual Output & \\
  \hline
  Conclusion & \\
  \hline
\end{tabular}
\end{center}           
\end{table}

\begin{table}[h]
\begin{center}
\begin{tabular}{|l | l|}
\hline
  Test Number & \\
  \hline
  Requirement Reference & \\
  \hline
  Requirement &  \\
  \hline
  Input & \\
  \hline
  Desired Output & \\
  \hline
  Actual Output & \\
  \hline
  Conclusion & \\
  \hline
\end{tabular}
\end{center}           
\end{table}

\begin{table}[h]
\begin{center}
\begin{tabular}{|l | l|}
\hline
  Test Number & \\
  \hline
  Requirement Reference & \\
  \hline
  Requirement &  \\
  \hline
  Input & \\
  \hline
  Desired Output & \\
  \hline
  Actual Output & \\
  \hline
  Conclusion & \\
  \hline
\end{tabular}
\end{center}           
\end{table}

\section{Nonfunctional Requirements Evaluation}

\subsection{Usability}
		
\subsection{Performance}

\subsection{etc.}
	
\section{Comparison to Existing Implementation}	

This section will not be appropriate for every project.

\section{Unit Testing}

\section{Changes Due to Testing}

\section{Automated Testing}
		
\section{Trace to Requirements}
		
\section{Trace to Modules}		

\section{Code Coverage Metrics}

\bibliographystyle{plainnat}
\bibliography{../../refs/References}

\newpage{}
\section*{Appendix --- Reflection}

The information in this section will be used to evaluate the team members on the
graduate attribute of Lifelong Learning.  Please answer the following questions:

\begin{enumerate}
  \item 
  \item 
\end{enumerate}

\end{document}