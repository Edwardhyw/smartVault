\documentclass[12pt, titlepage]{article}

\usepackage{booktabs}
\usepackage{tabularx}
\usepackage{hyperref}
\usepackage{float}
\hypersetup{
    colorlinks,
    citecolor=black,
    filecolor=black,
    linkcolor=red,
    urlcolor=blue
}
\usepackage[round]{natbib}

%% Comments

\usepackage{color}

\newif\ifcomments\commentstrue %displays comments
%\newif\ifcomments\commentsfalse %so that comments do not display

\ifcomments
\newcommand{\authornote}[3]{\textcolor{#1}{[#3 ---#2]}}
\newcommand{\todo}[1]{\textcolor{red}{[TODO: #1]}}
\else
\newcommand{\authornote}[3]{}
\newcommand{\todo}[1]{}
\fi

\newcommand{\wss}[1]{\authornote{blue}{SS}{#1}} 
\newcommand{\plt}[1]{\authornote{magenta}{TPLT}{#1}} %For explanation of the template
\newcommand{\an}[1]{\authornote{cyan}{Author}{#1}}

%% Common Parts

\newcommand{\progname}{Mechtronics Enigeering} % PUT YOUR PROGRAM NAME HERE
\newcommand{\authname}{Team 32, Wingman
\\ Edward He
\\ Erping Zhang
\\ Guangwei Tang
\\ Peng Cui
\\ Peihua Jin } % AUTHOR NAMES                  

\usepackage{hyperref}
    \hypersetup{colorlinks=true, linkcolor=blue, citecolor=blue, filecolor=blue,
                urlcolor=blue, unicode=false}
    \urlstyle{same}
                                


\begin{document}

\title{Verification and Validation Report: \progname} 
\author{\authname}
\date{\today}
	
\maketitle

\pagenumbering{roman}

\section{Revision History}

\begin{tabularx}{\textwidth}{p{3cm}p{2cm}X}
\toprule {\bf Date} & {\bf Version} & {\bf Notes}\\
\midrule
Date 1 & 1.0 & Notes\\
Date 2 & 1.1 & Notes\\
\bottomrule
\end{tabularx}

~\newpage

\section{Symbols, Abbreviations and Acronyms}

\renewcommand{\arraystretch}{1.2}
\begin{tabular}{l l} 
  \toprule		
  \textbf{symbol} & \textbf{description}\\
  \midrule 
  T & Test\\
  \bottomrule
\end{tabular}\\

\wss{symbols, abbreviations or acronyms -- you can reference the SRS tables if needed}

\newpage

\tableofcontents

\listoftables %if appropriate

\listoffigures %if appropriate

\newpage

\pagenumbering{arabic}

This document ...

\section{Functional Requirements Evaluation}
\begin{table}[H]
\begin{center}
\begin{tabular}{|p{0.25\textwidth} | p{0.75\textwidth}|}
\hline
  Test Number & IPR1-1\\
  \hline
  Requirement Reference & IPR1\\
  \hline
  Requirement &  The system should be able to identify human’s body\\
  \hline
  Input & Images of the working environment and a human show up in
the environment\\
  \hline
  Desired Output & Coordinate of the detected human body\\
  \hline
  Actual Output & Correct coordinate of the detected human body\\
  \hline
  Conclusion & The test pass as expected\\
  \hline
\end{tabular}
\end{center}           
\end{table}

\begin{table}[H]
\begin{center}
\begin{tabular}{|p{0.25\textwidth} | p{0.75\textwidth}|}
\hline
  Test Number & UIR4-1\\
  \hline
  Requirement Reference & UIR4\\
  \hline
  Requirement &  The UI must be able to allow the user to view the system’s status at any given point in time.\\
  \hline
  Input & User change the unplug the camera to insert a fault\\
  \hline
  Desired Output & The graphical display to the user\\
  \hline
  Actual Output & \\
  \hline
  Conclusion & \\
  \hline
\end{tabular}
\end{center}           
\end{table}



\section{Nonfunctional Requirements Evaluation}

\subsection{Usability}
\begin{table}[H]
\begin{center}
\begin{tabular}{|p{0.25\textwidth} | p{0.75\textwidth}|}
\hline
  Test Number &  APR1-1 \\
  \hline
  Requirement Reference & ARP1,APR2,SCR1 \\
  \hline
  Requirement & No electronic components should be visible and exposed. The mount should stay still without any physical changes  \\
  \hline
  Input & Launch the program normally and give the camera mount a physical impact \\
  \hline
  Desired Output & The mount should not be broken and there should not be any visible dislocation of any parts\\
  \hline
  Actual Output & The mount undergoes a planar movement. No visible parts broken or dislocation. The arduino board attached at the bottom stays still  \\
  \hline
  Conclusion & The test pass as expected\\
  \hline
\end{tabular}
\end{center}           
\end{table}
\\
\begin{table}[H]
\begin{center}
\begin{tabular}{|p{0.25\textwidth} | p{0.75\textwidth}|}
\hline
  Test Number & EUR1-1\\
  \hline
  Requirement Reference & EUR1,LER1,LER2\\
  \hline
  Requirement &  Users without electronics and coding background will be able to connect the hardware and use the program\\
  \hline
  Input & Users are asked to connect the hardware and start the program\\
  \hline
  Desired Output & There should not be any unclear instructions for the user to proceed. The hardware system including the Arduino board, camera and mount should be clarified for people to plug the wires\\
  \hline
  Actual Output & As camera,Arduino board and the motor are already attached to the mount. User just need to plug the wires to corresponding pins then they can simply start the program with one click \\
  \hline
  Conclusion & The test pass as expected\\
  \hline
\end{tabular}
\end{center}           
\end{table}
\subsection{Performance}
\begin{table}[H]
\begin{center}
\begin{tabular}{|p{0.25\textwidth} | p{0.75\textwidth}|}
\hline
  Test Number & SCR3-1\\
  \hline
  Requirement Reference & SCR3\\
  \hline
  Requirement &  Rotation speed of the camera should be appropriate and will not damage other parts under the condition the camera have to rotate from one end to the other\\
  \hline
  Input & Human walk through the camera and leave the capture region at high pace\\
  \hline
  Desired Output & The camera will detect the human body and starts to follow the human movement. Once the human accelerate and leave the region, the camera will stop tracking and the rotation speed will not be fast enough to damage other parts\\
  \hline
  Actual Output & The camera will rotate to the human position and follow the movement once it detects the existence of human body. As the human quickly leave the capture region, the camera stops tracking and take a photo of the current frame. After 2 seconds, it will rotate back to the original position. There are no parts being damaged during the movement \\
  \hline
  Conclusion & The test pass as expected\\
  \hline
\end{tabular}
\end{center}           
\end{table}
\subsection{etc.}
	
\section{Comparison to Existing Implementation}	

This section will not be appropriate for every project.

\section{Unit Testing}

\section{Changes Due to Testing}

\section{Automated Testing}
		
\section{Trace to Requirements}
		
\section{Trace to Modules}		

\section{Code Coverage Metrics}

\bibliographystyle{plainnat}
\bibliography{../../refs/References}

\newpage{}
\section*{Appendix --- Reflection}

The information in this section will be used to evaluate the team members on the
graduate attribute of Lifelong Learning.  Please answer the following questions:

\begin{enumerate}
  \item 
  \item 
\end{enumerate}

\end{document}
